\chapter{Аналитическая часть}

В этом разделе будет представлена информация по поводу графа синтаксических зависимостей в тексте.

\section[Граф синтаксических зависимостей текста]{Граф синтаксических зависимостей\\текста}

Дерево синтаксических зависимостей предложения представляет собой структуру, которая отображает связи между словами в предложении и их синтаксические роли.

В таком графе каждое слово представлено узлом, а связи между словами представлены направленными ребрами. 
Каждое ребро указывает зависимость между двумя словами, где одно слово является зависимым (дочерним), а другое - главным (родительским).

Синтаксическое дерево помогает визуализировать структуру предложения и понять связи между словами. 
Оно может быть использовано для различных задач обработки естественного языка, таких как синтаксический анализ, машинный перевод, именованное сущностное распознавание и другие.

Граф синтаксических зависимостей текста строится на основе деревьев синтаксических зависимостей предложений. 
В нем каждое слово представлено узлом (возможно, с дополнительной информацией, такой как лемма или часть речи), а связи между словами являются направленными ребрами, причем ребра помечаются количеством таких связей между соответствующими словами.

Библиотека \textit{spacy} предоставляет инструменты для анализа синтаксических зависимостей и визуализации деревьев синтаксических зависимостей, которые помогают исследователям и разработчикам в понимании структуры текста и автоматическом анализе естественного языка.

А библиотека \textit{pymorphy2} предоставляет функции для получения информации о словах, такой как лемма (нормальная форма слова), грамматическая информация, часть речи и т.д. 
Она основана на морфологической разметке и обучена на большом количестве русских текстов.

\section*{Вывод}

В данном разделе было рассмотрено понятие графа синтаксических зависимостей.