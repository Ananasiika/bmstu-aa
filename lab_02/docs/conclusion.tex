\chapter*{Заключение}
\addcontentsline{toc}{chapter}{Заключение}

В результате эксперимента было получено, что при размерах матрицы свыше 10, оптимизированный алгоритм Винограда быстрее обычного алгоритма Винограда более, чем 1.2 раза, быстрее алгоритма Штрассена примерно в 20 раз, а так же быстрее стандартного алгоритма в 1.1 раз. 

Также при проведении эксперимента было выявлено, что на четных размерах реализация алгоритма Винограда в 1.2 раза быстрее, чем на нечетных размерах матриц, что обусловлено необходимостью проводить дополнительные вычисления для крайних строк и столбцов.  

В итоге, можно сказать, что при матрицы более 10 следует использовать оптимизированный алгоритм Винограда, а при меньших размерах - стандартный алгоритм.

При рассмотрении трудоемкости алгоритмов были получены следующие результаты: алгоритм Штрассена имеет наименьшую трудоемкость при матрицах, размер которых более 100 и является степенью двойки, в остальных случаях выигрывает оптимизированный алгоритм Винограда, после которого идет стандартный алгоритм, а самым трудоемким оказался обычный алгоритм Винограда.

В ходе выполнения лабораторной работы были описаны и реализованы алгоритмы умножения матриц: классический, Винограда, его оптимизация и Штрассена, проведено тестирование для измерения времени выполнения и использования помяти для каждого алгоритма, а также проведен сравнительный анализ процессорного времени приведенных алгоритмов и подготовлен отчет о лабораторной работе.

Поставленная цель лабораторной работы была достигнута.