\begin{center}
	ЗАКЛЮЧЕНИЕ
\end{center}
\addcontentsline{toc}{chapter}{ЗАКЛЮЧЕНИЕ}

В результате эксперимента было получено, что при размерах матрицы свыше 10, оптимизированный алгоритм Винограда быстрее обычного алгоритма Винограда более, чем 1.2 раза, быстрее алгоритма Штрассена примерно в 20 раз, а так же быстрее стандартного алгоритма в 1.1 раз. В итоге, можно сказать, что при таких данных следует использовать оптимизированный алгоритм Винограда.

Также при проведении эксперимента было выявлено, что на четных размерах реализация алгоритма Винограда в 1.2 раза быстрее, чем на нечетных размерах матриц, что обусловлено необходимостью проводить дополнительные вычисления для крайних строк и столбцов.  Следовательно, стоит использовать алгоритм Винограда для матриц, которые имеют четные размеры.

При рассмотрении трудоемкости алгоритмов были получены следующие результаты: алгоритм Штрассена имеет наименьшую трудоемкость при матрицах, размер которых более 100 и является степенью двойки, в остальных случаях выигрывает оптимизированный алгоритм Винограда, после которого идет стандартный алгоритм, а самым трудоемким оказался обычный алгоритм Винограда.

В ходе выполнения лабораторной работы были решены следующие задачи:
\begin{itemize}
	\item были изучены и реализованы алгоритмы умножения матриц:  классический, Винограда, его оптимизацию и Штрассена;
    \item проведено тестирование для измерения времени выполнения и использования помяти для каждого алгоритма;
    \item проведен сравнительный анализ процессорного времени приведенных алгоритмов;
	\item подготовлен отчет о лабораторной работе.
\end{itemize}

Поставленная цель лабораторной работы была достигнута.