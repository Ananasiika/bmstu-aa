\chapter{Исследовательская часть}

В данном разделе будут приведены примеры работы программы, а также проведен сравнительный анализ реализаций алгоритмов при различных ситуациях на основе полученных данных.

\section{Технические характеристики}

Технические характеристики устройства, на котором выполнялся эксперимент, представлены далее:

\begin{itemize}
    \item операционная система - Ubuntu 22.04.3 \cite{ubuntu} Linux \cite{linux} x86\_64;
    \item память - 16 Гб;
    \item процессор - Intel® Core™ i5-1135G7 @ 2.40ГГц.
\end{itemize}

При эксперименте ноутбук не был включен в сеть электропитания.

\section{Демонстрация работы программы}

На рисунке \ref{img:example} представлен результат работы программы, на котором выводится меню, выполняется умножение матриц $\begin{pmatrix}
	2 & 3 & 4 \\
	3 & 4 & 5 \\
\end{pmatrix}$ и $\begin{pmatrix}
3 & 4 & 5 & 6 \\
5 & 4 & 3 & 2 \\
2 & 3 & 6 & 6 \\
\end{pmatrix}$ всеми 4 алгоритмами и результат выводится на экран.

\imgHeight{190mm}{example}{Пример работы программы}
\clearpage

\section{Время выполнения реализации алгоритмов}

Как было сказано выше, используется функция замера процессорного времени process\_time(...) из библиотеки time на Python. Функция возвращает пользовательское процессорное время типа float.

Использовать функцию приходится дважды, затем из конечного времени нужно вычесть начальное, чтобы получить результат.

Замеры проводились для матриц размером от 2 до 75 по 100 раз на различных входных матрицах.

Результаты замеров приведены в таблице \ref{tbl:time_mes} (время в мкс).

\begin{table}[h]
    \begin{center}
        \begin{threeparttable}
        \captionsetup{justification=raggedright,singlelinecheck=off}
        \caption{Результаты замеров времени}
        \label{tbl:time_mes}
        \begin{tabular}{|c|c|c|c|c|}
            \hline
            Размер & Стандартный & Виноград & Виноград (опт) & Штрассен \\ 
            \hline
            2 & 8218.84 & 12583.32 & 13206.95 & 44853.92 \\ 
            \hline
            4 & 12207.12 & 16312.57 & 15636.80 & 140016.41 \\ 
            \hline
            8 & 61371.23 & 80948.35 & 74771.01 & 964163.95 \\ 
            \hline
            10 & 114888.44 & 147519.98 & 126668.54 & 6695137.08 \\ 
            \hline
            15 & 392224.91 & 447195.45 & 367630.54 & 6730375.88 \\ 
            \hline
            16 & 451583.92 & 517977.17 & 446052.49 & 6794859.96 \\ 
            \hline
            20 & 884854.95 & 998077.34 & 864776.02 & 50403825.46 \\ 
            \hline
            25 & 1733791.32 & 1929171.57 & 1599970.97 & 50669717.92 \\ 
            \hline
            30 & 2992857.34 & 3191459.81 & 2661066.53 & 49970742.91 \\ 
            \hline
            32 & 3410088.33 & 3680717.81 & 3153943.64 & 49520702.65 \\ 
            \hline
            35 & 4684638.14 & 5025734.80 & 4324064.92 & 350374351.26 \\ 
            \hline
            40 & 7043207.35 & 7541179.09 & 6344228.52 & 361474685.14 \\ 
            \hline
            45 & 9936251.45 & 10389906.65 & 8707965.21 & 357458368.76 \\ 
            \hline
            50 & 13718930.74 & 14088592.20 & 11780340.67 & 393463534.63 \\ 
            \hline
            55 & 20068585.48 & 21020754.80 & 18465829.43 & 412183255.14 \\ 
            \hline
            60 & 25700641.25 & 26571696.68 & 22441666.79 & 399229563.90 \\ 
            \hline
            64 & 31508333.87 & 33101892.75 & 28246602.11 & 403796660.20 \\ 
            \hline
            65 & 32977113.41 & 34376070.33 & 28565568.59 & 2675449374.09 \\ 
            \hline
            70 & 40812887.65 & 41532495.83 & 34562454.85 & 2780235588.35 \\ 
            \hline
            75 & 49781888.03 & 51604794.53 & 44260030.02 & 2595885483.53 \\ 
            \hline
		\end{tabular}
    \end{threeparttable}
\end{center}
\end{table}

\clearpage
Также на рисунках \ref{img:graph_all}--\ref{img:graph} приведены графические результаты замеров.

\imgHeight{100mm}{graph_all}{Сравнение по времени реализаций алгоритмов: стандартного, Винограда, оптимизированного Винограда и Штрассена }
\imgHeight{100mm}{graph}{Сравнение по времени реализаций алгоритма Винограда и оптимизированного алгоритма Винограда}
\clearpage


\section{Вывод}

В результате эксперимента было получено, что при размерах матрицы свыше 10, оптимизированный алгоритм Винограда быстрее обычного алгоритма Винограда более, чем 1.2 раза, быстрее алгоритма Штрассена примерно в 20 раз, а так же быстрее стандартного алгоритма в 1.1 раз. 

Также при проведении эксперимента было выявлено, что на четных размерах реализация алгоритма Винограда в 1.2 раза быстрее, чем на нечетных размерах матриц, что обусловлено необходимостью проводить дополнительные вычисления для крайних строк и столбцов.  

В итоге, можно сказать, что при размерах матрицы более 10 следует использовать оптимизированный алгоритм Винограда, а при меньших размерах - стандартный алгоритм.