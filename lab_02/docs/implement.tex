\chapter{Технологическая часть}

В данном разделе будут рассмотрены средства реализации, используемые типы данных, а также представлены листинги алгоритмов умножения матриц.

\section{Средства реализации}
В данной работе для реализации был выбран язык программирования $Python \cite{python-lang}$. В текущей лабораторной работе требуется замерить процессорное время для выполняемой программы, а также построить графики. Все эти инструменты присутствуют в выбранном языке программирования.

Время работы было замерено с помощью функции \textit{process\_time(...)} из библиотеки $time \cite{python-lang-time}$.

\section{Описание используемых типов данных}

При реализации алгоритмов будут использованы следующие типы данных:

\begin{itemize}
	\item количество строк - целое число типа \textit{int}
	\item количество столбцов - целое число типа \textit{int}
	\item матрица - двумерный список типа \textit{int}
\end{itemize}



\section{Реализация алгоритмов}

В листингах \ref{lst:stand_alg}-\ref{lst:vin_alg} представлены реализации алгоритмов умножения матриц - стандартного, Винограда, оптимизированного алгоритма Винограда. Реализация алгоритма Штрассена находится в приложении.

\begin{center}
    \captionsetup{justification=raggedright,singlelinecheck=off}
    \begin{lstlisting}[label=lst:stand_alg,caption=Стандартный алгоритм умножения матриц]
def standart_alg(mat_a, mat_b):
	rows_a = len(mat_a)
	cols_a = len(mat_a[0])
	cols_b = len(mat_b[0])
	if cols_a != len(mat_b):
		return []
	result = [[0] * cols_b for _ in range(rows_a)]
	for i in range(rows_a):
		for j in range(cols_b):
			for k in range(cols_a):
				result[i][j] += mat_a[i][k] * mat_b[k][j]
	return result
	\end{lstlisting}
	\captionsetup{justification=raggedright,singlelinecheck=off}
	\begin{lstlisting}[label=lst:opt_vin_alg,caption=Оптимизированный алгоритм Винограда]
def optimized_vinograd_alg(mat_a, mat_b):
	rows_a = len(mat_a)
	cols_a = len(mat_a[0])
	rows_b = len(mat_b)
	cols_b = len(mat_b[0])
	if cols_a != rows_b:
		return []
	result = [[0] * cols_b for _ in range(rows_a)]
	mul_row = [0] * rows_a
	mul_col = [0] * cols_b
	for i in range(rows_a):
		for j in range(cols_a >> 1):
			mul_row[i] += mat_a[i][j << 1] * mat_a[i][(j << 1) + 1]
	for i in range(cols_b):
		for j in range(rows_b >> 1):
			mul_col[i] += mat_b[j << 1][i] * mat_b[(j << 1) + 1][i]
	flag = cols_a % 2
	for i in range(rows_a):
		for j in range(cols_b):
			result[i][j] = -mul_row[i] - mul_col[j]
	for k in range(1, cols_a, 2):
		result[i][j] += (mat_a[i][k - 1] + mat_b[k][j]) * (mat_a[i][k] + mat_b[k - 1][j])
	if flag:
		result[i][j] += mat_a[i][cols_a - 1] * mat_b[rows_b - 1][j]
	return result
	\end{lstlisting}
\end{center}


\begin{center}
    \captionsetup{justification=raggedright,singlelinecheck=off}
    \begin{lstlisting}[label=lst:vin_alg,caption=Алгоритм Винограда]
def vinograd_alg(mat_a, mat_b):
	rows_a = len(mat_a)
	cols_a = len(mat_a[0])
	rows_b = len(mat_b)
	cols_b = len(mat_b[0])
	if cols_a != rows_b:
		return []
	result = [[0] * cols_b for _ in range(rows_a)]
	mul_row = [0] * rows_a
	mul_col = [0] * cols_b
	for i in range(rows_a):
		for j in range(cols_a // 2):
			mul_row[i] = mul_row[i] + mat_a[i][2 * j] * mat_a[i][2 * j + 1]
	for i in range(cols_b):
		for j in range(rows_b // 2):
			mul_col[i] = mul_col[i] + mat_b[2 * j][i] * mat_b[2 * j + 1][i]
	for i in range(rows_a):
		for j in range(cols_b):
			result[i][j] = -mul_row[i] - mul_col[j]
			for k in range(cols_a // 2):
				result[i][j] = result[i][j] + (mat_a[i][2 * k] + mat_b[2 * k + 1][j]) * (mat_a[i][2 * k + 1] + mat_b[2 * k][j])
			if cols_a % 2 == 1:
				result[i][j] = result[i][j] + mat_a[i][cols_a - 1] * mat_b[rows_b - 1][j]
	return result
\end{lstlisting}
\end{center}


\section{Функциональные тесты}

В таблице \ref{tbl:functional_test} приведены тесты для функций, реализующих алгоритмы умножения матриц, рассматриваемых в данной лабораторной работе. Тесты \textit{для всех алгоритмов} пройдены успешно.


\begin{table}[h]
	\begin{center}
		\begin{threeparttable}
		\captionsetup{justification=raggedright,singlelinecheck=off}
		\caption{\label{tbl:functional_test} Функциональные тесты}
		\begin{tabular}{|c@{\hspace{7mm}}|c@{\hspace{7mm}}|c@{\hspace{7mm}}|c@{\hspace{7mm}}|c@{\hspace{7mm}}|c@{\hspace{7mm}}|}
			\hline
			Матрица 1 & Матрица 2 & Ожидаемый результат \\ 
			\hline

			$\begin{pmatrix}
				1 & 5 & 7\\
				2 & 6 & 8\\
				3 & 7 & 9
			\end{pmatrix}$ &
			$\begin{pmatrix}
				&
			\end{pmatrix}$ &
			Сообщение об ошибке \\ \hline

			$\begin{pmatrix}
				1 & 2 & 3
			\end{pmatrix}$ &
			$\begin{pmatrix}
				1 & 2 & 3
			\end{pmatrix}$ &
			Сообщение об ошибке \\ \hline

			$\begin{pmatrix}
				1 & 1 & 1
			\end{pmatrix}$ &
			$\begin{pmatrix}
				1 \\
				1 \\
				1
			\end{pmatrix}$ &
			$\begin{pmatrix}
				1 & 1 & 1\\
				1 & 1 & 1 \\
				1 & 1 & 1
			\end{pmatrix}$ \\ \hline

			$\begin{pmatrix}
				1 & 2 & 3 \\
				4 & 5 & 6 \\
				7 & 8 & 9
			\end{pmatrix}$ &
			$\begin{pmatrix}
				1 & 0 & 0 \\
				0 & 1 & 0 \\
				0 & 0 & 1
			\end{pmatrix}$ &
			$\begin{pmatrix}
				1 & 2 & 3 \\
				4 & 5 & 6 \\
				7 & 8 & 9
			\end{pmatrix}$ \\ \hline

			$\begin{pmatrix}
				2
			\end{pmatrix}$ &
			$\begin{pmatrix}
				2
			\end{pmatrix}$ &
			$\begin{pmatrix}
				4
			\end{pmatrix}$ \\ \hline

		\end{tabular}
		\end{threeparttable}
	\end{center}
	
\end{table}

\section{Вывод}

Были представлены листинги всех алгоритмов умножения матриц - стандартного, Винограда, оптимизированного алгоритма Винограда и Штрассена. Также в данном разделе была приведена информация о выбранных средствах для разработки алгоритмов и используемых типов данных.
