\chapter{Конструкторская часть}
В этом разделе будут представлены схемы алгоритмов перемножения матриц - стандартного, Винограда, оптимизации алгоритма Винограда и \\Штрассена.

\section{Сведения о модулях программы}
Программа состоит из двух модулей:
\begin{itemize}
	\item $main.py$ - файл, содержащий весь служебный код;
    \item $algorythms.py$ - файл, содержащий код всех алгоритмов перемножения матриц. \newline
\end{itemize}


\section{Разработка алгоритмов}
На рисунке \ref{img:stand_alg} представлена схема алгоритма для стандартного умножения матриц. На рисунках \ref{img:vin_alg1}-\ref{img:vin_alg2} схема алгоритма Винограда умножения матриц, на \ref{img:opt_vin_alg1}-\ref{img:opt_vin_alg2} схема оптимизированного алгоритма Винограда, а на рисунках \ref{img:strassen}-\ref{img:strassen2} схема алгоритма Штрассена. 

\imgScale{0.75}{stand_alg}{Схема стандартного алгоритма умножения матриц}
\imgScale{0.6}{vin_alg1}{Схема алгоритма Винограда (часть 1)}
\imgScale{0.6}{vin_alg2}{Схема алгоритма Винограда (часть 2)}
\imgScale{0.6}{opt_vin_alg1}{Схема оптимизированного алгоритма Винограда (часть 1)}
\imgScale{0.6}{opt_vin_alg2}{Схема оптимизированного алгоритма Винограда (часть 2)}
\imgScale{0.6}{strassen}{Схема алгоритма Штрассена (часть 1)}
\imgScale{0.6}{strassen2}{Схема алгоритма Штрассена (часть 2)}

\clearpage


\section{Модель вычислений}

Чтобы провести вычисление трудоемкости алгоритмов умножения матриц, введем модель вычислений \cite{model}:

\begin{enumerate}
	\item Трудоемкость следующих базовых операций единична:
	+, -, =, +=, -=, ==, !=, <, >, <=, >=, [], ++, --, <<, >>.
	Операции *, \%, / имеют трудоемкость 2.
	\item трудоемкость оператора выбора \code{if условие then A else B} рассчитывается, как (\ref{for:if});
	\begin{equation}
		\label{for:if}
		f_{if} = f_{\text{условия}} +
		\begin{cases}
			f_A, & \text{если условие выполняется,}\\
			f_B, & \text{иначе.}
		\end{cases}
	\end{equation}
	\item трудоемкость цикла рассчитывается, как (\ref{for:for});
	\begin{equation}
		\label{for:for}
		f_{for} = f_{\text{инициализации}} + f_{\text{сравнения}} + N(f_{\text{тела}} + f_{\text{инкремента}} + f_{\text{сравнения}})
	\end{equation}
	\item трудоемкость передачи параметров в функцию и возврат из нее равны 0.
\end{enumerate}


\section{Трудоемкость алгоритмов}

Рассчитаем трудоемкость алгоритмов умножения матриц. Введем обозначения: $M =$ количество строк матрицы $А$, $N -$ количество столбцов $B$, $K -$ количество столбцов $А$ или количество строк $B$.

\subsection{Стандартный алгоритм умножения матриц}

Для стандартного алгоритма умножения матриц трудоемкость будет складываться из:

\begin{itemize}
	\item внешнего цикла по $i \in [1..M]$, трудоёмкость которого: $f = 2 + M \cdot (2 + f_{body})$;
	\item цикла по $j \in [1..N]$, трудоёмкость которого: $f = 2 + N \cdot (2 + f_{body})$;
	\item цикла по $k \in [1..K]$, трудоёмкость которого: $f = 2 + 9K$. \newline
\end{itemize}

Поскольку трудоемкость стандартного алгоритма равна трудоемкости внешнего цикла, то:

\begin{equation}
	\label{for:standard}
	f_{standard} = 2 + M \cdot (4 + N \cdot (4 + 9K)) = 2 + 4M + 4MN + 9MNK \approx 9MNK
\end{equation}


\subsection{Алгоритм Винограда}

Чтобы вычислить трудоемкость алгоритма Винограда, нужно учесть следующее: 

\begin{itemize}
	\item создание и инициализация массивов a\_tmp и b\_tmp, трудоёмкость которых (\ref{for:init}):
	\begin{equation}
		\label{for:init}
		f_{init} = M + N;
	\end{equation}
	
	\item заполнение массива a\_tmp, трудоёмкость которого (\ref{for:ATMP}):
	\begin{equation}
		\label{for:ATMP}
		f_{a\_tmp} = 2 + M (4 + \frac{K}{2} \cdot 15);
	\end{equation}
	
	\item заполнение массива b\_tmp, трудоёмкость которого (\ref{for:BTMP}):
	\begin{equation}
		\label{for:BTMP}
		f_{b\_tmp} = 2 + N (4 + \frac{K}{2} \cdot 15);
	\end{equation}
	
	\item цикл заполнения для чётных размеров, трудоёмкость которого (\ref{for:cycle}):
	\begin{equation}
		\label{for:cycle}
		f_{cycle} = 2 + M (4 + N \cdot (14 + \frac{K}{2} \cdot 28));
	\end{equation}
	
	\item цикл заполнения для нечётных размеров, трудоемкость которого (\ref{for:cycle_odd}):
	\begin{equation}
		\label{for:cycle_odd}
		f_{cycle} = 2 + M (4 + N \cdot (28 + \frac{K}{2} \cdot 28)).
	\end{equation}
\end{itemize}

Тогда для худшего случая (нечётный общий размер матриц) имеем (\ref{for:bad}):
\begin{equation}
	\label{for:bad}
	f_{worst} =  f_{a\_tmp} + f_{b\_tmp} + f_{cycle_odd}\approx 14 \cdot MNK
\end{equation}

Для лучшего случая (чётный общий размер матриц) имеем (\ref{for:good}):
\begin{equation}
	\label{for:good}
f_{best} =  f_{a\_tmp} + f_{a\_tmp} + f_{cycle} \approx 14 \cdot MNK
\end{equation}


\subsection{Оптимизированный алгоритм Винограда}

Оптимизация заключается в:
\begin{itemize}
    \item использовании побитового сдвига вместо деления на 2;
    \item операции сложения и вычитания заменены на операции $+=$ и $-=$ соответственно;
    \item вычисление четности матрицы вынесено из цикла.
    \newline
\end{itemize}

Тогда трудоемкость оптимизированного алгоритма Винограда состоит из:

\begin{itemize}
	\item создания и инициализации массивов a\_tmp и b\_tmp (\ref{for:init});
	
	\item заполнения массива a\_tmp, трудоёмкость которого (\ref{for:ATMP});
	
	\item заполнения массива b\_tmp, трудоёмкость которого (\ref{for:BTMP});
	
	\item цикла заполнения для чётных размеров, трудоёмкость которого (\ref{for:impr_cycle}):
	\begin{equation}
		\label{for:impr_cycle}
		f_{cycle} = 2 + M (4 + N \cdot (11 + \frac{K}{2} \cdot 17));
	\end{equation}
	
	\item цикла заполнения для чётных размеров, трудоёмкость которого (\ref{for:impr_cycle_odd}):
	\begin{equation}
		\label{for:impr_cycle_odd}
		f_{cycle} = 2 + M (4 + N \cdot (22 + \frac{K}{2} \cdot 17)).
	\end{equation}
\end{itemize}

Тогда для худшего случая (нечётный общий размер матриц) имеем (\ref{for:bad_impr}):
\begin{equation}
	\label{for:bad_impr}
	f_{worst} =  f_{a\_tmp} + f_{a\_tmp} + f_{impr_cycle_odd} \approx 8.5MNK
\end{equation}

Для лучшего случая (чётный общий размер матриц) имеем (\ref{for:good_impr}):
\begin{equation}
	\label{for:good_impr}
	f_{best} = f_{a\_tmp} + f_{a\_tmp} + f_{impr_cycle} \approx 8.5MNK
\end{equation}


\subsection{Алгоритм Штрассена}

Введем размер S - ближайшая степень двойки, которая больше или равна размерам матриц A и B.


Для алгоритма Штрассена умножения матриц трудоемкость будет складываться из:

\begin{itemize}
	\item создания и инициализации матриц a\_new и b\_new, трудоемкость которого (\ref{for:init_s}):
	\begin{equation}
		\label{for:init_s}
		f_{init} = f_{a\_new} + f_{b\_new} = 2 * S;
	\end{equation}
	\item заполнения матрицы a\_new, трудоемкость которого (\ref{for:a_new}):
	\begin{equation}
		\label{for:a_new}
		f_{fill_a} = 2 + M (4 + K \cdot 5);
	\end{equation}
	\item заполнения матрицы b\_new, трудоемкость которого (\ref{for:b_new}):
	\begin{equation}
		\label{for:b_new}
		f_{fill_b} = 2 + K (4 + N \cdot 5);
	\end{equation}
	\item вызова функции $strassen\_recursive$, трудоемкость которого 0 и трудоемкости этой функции, которую посчитаем отдельно;
	\item заполнения матрицы C, трудоемкость которого (\ref{for:c_new}):
	\begin{equation}
		\label{for:c_new}
		f_{fill_c} = 2 + M \cdot 3.
	\end{equation}
	\newline
\end{itemize}

Вычислим трудоемкость функции $strassen\_recursive$. Пусть $n$ - размер матриц, которые передаются в эту функцию, тогда трудоемкость состоит из:
\begin{itemize}
	\item создания и инициализации матриц a11, a12, a21, a22, b11, b12, b21, b22, трудоемкость которого (\ref{for:init_s_r}):
	\begin{equation}
		\label{for:init_s_r}
		f_{init} = 8 \cdot (\frac{n}{2} + \frac{n}{2}) = 8n;
	\end{equation}
	\item заполнения этих матриц, трудоемкость которого (\ref{for:fill_r}):
	\begin{equation}
		\label{for:fill_r}
		f_{fill} = 16 + 20n;
	\end{equation}
	\item вызова $strassen\_recursive$ 7 раз, трудоемкость которых 0, но трудоемкость вычисления передаваемых параметров (\ref{for:calc}):
	\begin{equation}
		\label{for:calc}
		f_{calc\_param} = 10 \cdot (2 + n + 2n^2);
	\end{equation}
	\item вычисления подматриц результата, трудоемкость которого (\ref{for:calc_res}):
	\begin{equation}
		\label{for:calc_res}
		f_{calc\_res} = 8 \cdot (2 + n + 2n^2);
	\end{equation}
	\item составления результирующей матрицы, трудоемкость которого (\ref{for:res}):
	\begin{equation}
		\label{for:res}
		f_{res} = 4 + 4n.
	\end{equation}
\end{itemize}

Таким образом, общая трудоемкость алгоритма Штрассена (\ref{for:sht}):
\begin{equation}
	\label{for:sht}
	f_{sht} = 6 + 2S + 7M + 5MK + 5NK + \sum_{n=1}^{\frac{n}{2}}{52 + 28n + 36n^2}
\end{equation}

Сумма происходит для $n$ по степеням двойки от 1 до $\frac{n}{2}$.

\section{Классы эквивалентности при тестировании}

Для тестирования выделены классы эквивалентности, представленные ниже.

\begin{enumerate}
	\item Одна из матриц - пустая;
	\item Количество столбцов одной матрицы не равно количеству строк второй матрицы;
	\item Перемножение квадратных матриц;
	\item Перемножение матриц разных размеров (при этом количество столбцов одной матрицы не равно количеству строк второй матрицы).
\end{enumerate}


\section{Вывод}

В данном разделе были построены схемы алгоритмов умножения матриц рассматриваемых в лабораторной работе, были описаны классы эквивалентности для тестирования, модули программы, а также проведена теоретическая оценка трудоемкости алгоритмов.
