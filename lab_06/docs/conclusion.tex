\chapter*{Заключение}
\addcontentsline{toc}{chapter}{Заключение}

Результаты эксперимента показали, что при размере матрицы, равном 4, реализация муравьиного алгоритма работает в 184 раз медленнее, чем реализация алгоритма полного перебора. Однако, при размере матрицы, равном 9, реализация муравьиного алгоритма в 4 раза быстрее реализации алгоритма полного перебора, а при размере 10 --- уже в 31 раз. Следовательно, для матриц размером более 8, рекомендуется использовать муравьиный алгоритм, но стоит учесть возможные погрешности вычислений.

Также при исследовании было обнаружено, что чем меньше значение параметра $\alpha$, тем меньше возникает погрешностей. Кроме этого, количество дней (параметр $Days$) оказывает значительное влияние на отсутствие погрешностей. Чем больше значение $Days$, тем меньше погрешностей возникает.

В ходе лабораторной работы были выполнены следующие задачи: 
\begin{enumerate}[label=\arabic*)]
	\item описан муравьиный алгоритм и алгоритм полного перебора для решения задачи коммивояжера;
	\item реализованы эти алгоритмы;
	\item проведена параметризация муравьиного алгоритма и проведено сравнение времени работы реализаций муравьиного алгоритма и алгоритма полного перебора;
	\item описаны и обоснованы полученные результаты в отчете о выполненной лабораторной работе.
\end{enumerate}

Все задачи были решены, следовательно, поставленная цель лабораторной работы была достигнута.