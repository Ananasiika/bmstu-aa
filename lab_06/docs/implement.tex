\chapter{Технологическая часть}

В данном разделе будут рассмотрены средства реализации, а также представлены листинги алгоритмов полного перебора путей и муравьиного алгоритма.

\section{Средства реализации}
В данной работе для реализации был выбран язык программирования $Python~\cite{python-lang}$. В текущей лабораторной работе требуется замерить процессорное время для выполняемой программы, а также построить графики. Все эти инструменты присутствуют в выбранном языке программирования.

Время работы было замерено с помощью функции \textit{process\_time(...)} из библиотеки $time~\cite{python-lang-time}$.

\section{Описание используемых типов данных}

При реализации алгоритмов будут использованы следующие типы данных:

\begin{enumerate}[label=\arabic*)]
	\item размер матрицы смежности --- целое число типа \textit{int};
	\item название файла --- строка типа \textit{str};
	\item коэффициент $\alpha$, $\beta$, \textit{evaporation\_koef} --- числа типа \textit{float};
	\item коэффициент $e$ --- число типа \textit{int};
	\item матрица смежностей --- матрица типа \textit{int} для хранения длины путей между городами.
\end{enumerate}


\section{Листинги кода}

В листинге \ref{lst:full_comb} представлен алгоритм полного перебора путей, а в листингах \ref{lst:prob_arr}--\ref{lst:upd_pher_el} --- муравьиный алгоритм и дополнительные к нему функции.

\clearpage

\begin{center}
    \captionsetup{justification=raggedright,singlelinecheck=off}
    \begin{lstlisting}[label=lst:full_comb,caption=Алгоритм полного перебора путей]
def full_combinations(matrix, size):
	places = np.arange(size)
	places_combs = list()
	for combination in it.permutations(places):
		comb_arr = list(combination)
		places_combs.append(comb_arr)
	min_dist = float("inf")
	for i in range(len(places_combs)):
		cur_dist = 0
		for j in range(size - 1):
			start_city = places_combs[i][j]
			end_city = places_combs[i][j + 1]
			cur_dist += matrix[start_city][end_city]
		if (cur_dist < min_dist):
			min_dist = cur_dist
			best_way = places_combs[i]
	return min_dist, best_way
\end{lstlisting}
\end{center}

\begin{center}
\captionsetup{justification=raggedright,singlelinecheck=off}
\begin{lstlisting}[label=lst:prob_arr,caption=Алгоритм нахождения массива вероятностных переходов в непосещенные города]
def find_posibilyties(pheromones, visibility, visited, places, ant, alpha, beta):
	pk = [0] * places
	for place in range(places):
		if place not in visited[ant]:
			ant_place = visited[ant][-1]
			pk[place] = pow(pheromones[ant_place][place], alpha) * \
			pow(visibility[ant_place][place], beta)
		else:
			pk[place] = 0
	sum_pk = sum(pk)
	for place in range(places):
		pk[place] /= sum_pk  
	return pk
\end{lstlisting}
\end{center}
\clearpage


\begin{center}
    \captionsetup{justification=raggedright,singlelinecheck=off}
    \begin{lstlisting}[label=lst:ant_alg,caption=Муравьиный алгоритм]
def ant_algorythm(matrix, places, days, alpha, beta, k_evaporation, e):
	q = calc_q(matrix, places)
	best_way = []
	min_length = float("inf")
	pheromones = get_pheromones(places)
	visibility = get_visibility(matrix, places)
	ants = places
	for day in range(days):
		route = np.arange(places)
		visited = get_visited_places(route, ants)
		for ant in range(ants):
			while (len(visited[ant]) != ants):
				pk = find_posibilyties(pheromones, visibility, visited, places, ant, alpha, beta)  
				chosen_place = choose_next_place_by_posibility(pk)
				visited[ant].append(chosen_place - 1)
			cur_length = calc_length(matrix, visited[ant]) 
			if (cur_length < min_length):
				min_length = cur_length
				best_way = visited[ant]
		pheromones = update_pheromones(matrix, places, visited, pheromones, q, k_evaporation, e)
		pheromones = update_pheromones_elite(matrix, places, pheromones, q, e, best_way)
	return min_length, best_way
\end{lstlisting}
\end{center}


\begin{center}
    \captionsetup{justification=raggedright,singlelinecheck=off}
    \begin{lstlisting}[label=lst:choose_next,caption=Алгоритм нахождения следующего города на основании рандома]
def choose_next_place_by_posibility(pk):
	posibility = random()
	choice = 0
	chosen_place = 0
	while ((choice < posibility) and (chosen_place < len(pk))):
		choice += pk[chosen_place]
		chosen_place += 1
	return chosen_place
\end{lstlisting}
\end{center}

\begin{center}
    \captionsetup{justification=raggedright,singlelinecheck=off}
    \begin{lstlisting}[label=lst:upd_pher,caption=Алгоритм обновления матрицы феромонов для обычных муравьев]
def update_pheromones(matrix, places, visited, pheromones, q, k_evaporation):
	ants = places
	for i in range(places):
		for j in range(places):
			delta_pheromones = 0
			for ant in range(ants):
				length = calc_length(matrix, visited[ant])
				delta_pheromones += q / length
			pheromones[i][j] *= (1 - k_evaporation)
			pheromones[i][j] += delta_pheromones
			if (pheromones[i][j] < MIN_PHEROMONE):
				pheromones[i][j] = MIN_PHEROMONE
	return pheromones
\end{lstlisting}
\end{center}

\begin{center}
	\captionsetup{justification=raggedright,singlelinecheck=off}
	\begin{lstlisting}[label=lst:upd_pher_el,caption=Алгоритм обновления матрицы феромонов для элитных муравьев]
def update_pheromones_elite(matrix, places, pheromones, q, e, best_way):
	for i in range(places):
		for j in range(places):
			length = calc_length(matrix, best_way)
			delta_pheromones = e * q / length
			pheromones[i][j] += delta_pheromones
	return pheromones
	\end{lstlisting}
\end{center}

\section{Сведения о модулях программы}
Программа состоит из двух модулей:
\begin{enumerate}[label=\arabic*)]
	\item \textit{main.py} --- файл, содержащий меню программы, а также весь служебный код;
    \item \textit{algorythms.py} --- файл, содержащий реализацию алгоритма полного перебора и муравьиного алгоритма.
\end{enumerate}


\section{Функциональные тесты}

В таблице \ref{tbl:functional_test} приведены тесты для функций программы. Тесты \textit{для всех функций} пройдены успешно.

\begin{center}
    \captionsetup{justification=raggedright,singlelinecheck=off}
    \begin{longtable}[c]{|c|c|c|c|c|}
    \caption{Функциональные тесты\label{tbl:functional_test}} \\ \hline
		Матрица смежности & Ожидаемый результат & Результат программы \\
		\hline
		$ \begin{pmatrix}
			0 & 3 & 8 & 3 & 9 \\
			7 & 0 & 5 & 2 & 8 \\
			5 & 7 & 0 & 7 & 6 \\
			7 & 5 & 5 & 0 & 9 \\
			10 & 4 & 5 & 3 & 0 
		\end{pmatrix}$ &
		16, [0, 1, 3, 2, 4] &
		16, [0, 1, 3, 2, 4] \\
		\hline
		$ \begin{pmatrix}
			0 & 6 & 4 \\
			3 & 0 & 5 \\
			3 & 6 & 0	
		\end{pmatrix}$ &
		7, [1, 0, 2] &
		7, [1, 0, 2] \\
		\hline
		$ \begin{pmatrix}
			0 & 27 & 12 & 21 \\
			26 & 0 & 16 & 24 \\
			28 & 11 & 0 & 11 \\
			14 & 20 & 25 & 0 
		\end{pmatrix}$ &
		37, [3, 0, 2, 1] &
		37, [3, 0, 2, 1] \\
		\hline
	\end{longtable}
\end{center}

\section*{Вывод}

В данном разделе была приведена информация о выбранных средствах для реализации алгоритмов, листинги реализаций всех алгоритмов --- полного перебора и муравьиного и сведения о модулях программы, а так же проведено функциональное тестирование.
