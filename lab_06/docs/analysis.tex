\chapter{Аналитическая часть}
В этом разделе будет представлена информация о задаче коммивояжера, а также о путях ее решения --- полным перебором и муравьиным алгоритмом, а также информация о модификации муравьиного алгоритма с элитными муравьями.

\section{Задача коммивояжера}

\textbf{Задача коммивояжера} --- (задача о бродячем торговце) одна из самых важных задач транспортной логистики, в которой рассматриваются вершины графа, а также матрица смежности для расстояния между вершинами. Ее суть --- найти такой порядок посещения вершин графа, при котором путь будет минимален, а каждая вершина будет посещена только один раз~\cite{kom-task}.

\section{Алгоритм полного перебора для решения задачи коммивояжера}

\textbf{Полный перебор для задачи коммивояжера} --- имеет высокую сложность алгоритма ($n!$). Идея в полном переборе всех возможных путей в графе и выбор наименьшего из них. Решение будет получено, но потребуются большие затраты по времени выполнения даже при небольшом количестве вершин в графе~\cite{kom-task-all}.


\section{Муравьиный алгоритм}

\textbf{Муравьиный алгоритм} --- основан на принципе поведения колонии муравьев~\cite{ant-alg}.

Муравьи действуют, ощущая некий феромон. Каждый муравей, чтобы другие могли ориентироваться, оставляет на своем пути феромоны. При большом количестве прохождения муравьев, наибольшее количество феромона остается на оптимальном пути.

Суть в том, что муравей в одиночку ничего не может, поскольку способен выполнять только максимально простые задачи. Но при условии большого количества таких муравьев они могут самоорганизовываться для решения сложных задач.

Пусть муравей имеет следующие характеристики:
\begin{enumerate}[label=\arabic*)]
	\item зрение --- способен определить длину ребра;
	\item память --- запоминает пройденный маршрут;
	\item обоняние --- чувствует феромон.
\end{enumerate}

Также введем целевую функцию~\eqref{d_func}.

\begin{equation}
	\label{d_func}
	\eta_{ij} = 1 / D_{ij},
\end{equation}
где $D_{ij}$ — расстояние из текущего пункта $i$ до заданного пункта $j$.

А также понадобится формула вычисления вероятности перехода в заданную точку~\eqref{posib}.

\begin{equation}
	\label{posib}
	P_{kij} = \begin{cases}
		\frac{\tau_{ij}^a\eta_{ij}^b}{\sum_{q=1}^m \tau^a_{iq}\eta^b_{iq}}, \textrm{вершина не была посещена ранее муравьем k,} \\
		0, \textrm{иначе}
	\end{cases}
\end{equation}
где $a$ --- параметр влияния длины пути, $b$ --- параметр влияния феромона, $\tau_{ij}$ --- расстояния от города $i$ до $j$, $\eta_{ij}$ --- количество феромонов на ребре $ij$.

После завершения движения всех муравьев, формула обновляется феромон по формуле~\eqref{update_phero_1}:
\begin{equation}
	\label{update_phero_1}
		\tau_{ij}(t+1) = (1-p)\tau_{ij}(t) + \Delta \tau_{ij}.
\end{equation}
При этом
\begin{equation}
\label{update_phero_2}
 \Delta \tau_{ij} = \sum_{k=1}^N \tau^k_{ij},
\end{equation}
где
\begin{equation}
	\label{update_phero_3}
		 \Delta\tau^k_{ij} = \begin{cases}
		Q/L_{k}, \textrm{ребро посещено k-ым муравьем,} \\
		0, \textrm{иначе}
	\end{cases}
\end{equation}

Путь выбирается по следующей схеме:

\begin{enumerate}[label=\arabic*)]
	\item Каждый муравей имеет список запретов --- список уже посещенных городов (вершин графа).
	\item Муравьиной зрение --- отвечает за желание посетить вершину.
	\item Муравьиное обоняние --- отвечает за ощущение феромона на определенном пути (ребре). При этом количество феромона на пути (ребре) в момент времени $t$ обозначается как $\tau_{i, j} (t)$.
	\item После прохождения определенного ребра муравей откладывает на нем некотрое количество феромона, которое показывает оптимальность сделанного выбора (это количество вычисляется по формуле~\eqref{update_phero_3})
\end{enumerate}

\section{Вариация муравьиного алгоритма с элитными муравьями}

Из общего числа муравьев выделяются так называемые «элитные муравьи». По результатам каждой итерации алгоритма производится усиление лучших маршрутов путём прохода по данным маршрутам элитных муравьев и, таким образом, увеличение количества феромона на данных маршрутах. В такой системе количество элитных муравьев является дополнительным параметром, требующим определения. Так, для слишком большого числа элитных муравьев алгоритм может «застрять» на локальных экстремумах~\cite{ant-alg}.

\section*{Вывод}

В данном разделе была рассмотрена задача коммивояжера, а также полный перебор для ее решения и муравьиный алгоритм и его модификация с элитными муравьями.