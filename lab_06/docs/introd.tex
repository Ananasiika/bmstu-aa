\chapter*{Введение}
\addcontentsline{toc}{chapter}{Введение}

Лабораторная работа посвящена изучению одной из самых известных задач комбинаторной оптимизации --- проблеме коммивояжера. Проблема коммивояжера заключается в поиске оптимального пути для коммивояжера, который должен посетить заданное количество городов и пройти через каждый город ровно один раз, чтобы минимизировать общую длину пути.

\textbf{Целью данной работы} является описание муравьиного алгоритма, примененного к задаче коммивояжера. 
Для достижения поставленной цели необходимо выполнить следующие задачи:
\begin{enumerate}[label=\arabic*)]
	\item описать муравьиный алгоритм и алгоритм полного перебора для решения задачи коммивояжера;
	\item реализовать эти алгоритмы;
	\item провести параметризацию муравьиного алгоритма и сравнить время работы реализаций муравьиного алгоритма и алгоритма полного перебора;
    \item описать и обосновать полученные результаты в отчете о выполненной лабораторной работе.
\end{enumerate}
