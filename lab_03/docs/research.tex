\chapter{Исследовательская часть}

В данном разеделе будут приведены примеры работы программы, а также проведен сравнительный анализ алгоритмов при различных ситуациях на основе полученных данных.

\section{Технические характеристики}

Технические характеристики устройства, на котором выполнялся эксперимент представлены далее:

\begin{itemize}
	\item операционная система - Ubuntu 22.04.3 \cite{ubuntu} Linux \cite{linux} x86\_64;
	\item память - 16 Гб;
	\item процессор - Intel® Core™ i5-1135G7 @ 2.40ГГц.
\end{itemize}

При эксперименте ноутбук не был включен в сеть электропитания.

\section{Демонстрация работы программы}

На рисунке \ref{img:example} представлен результат работы программы, на котором выводится меню и выполняется каждая из 3 сортировок на предварительно подготовленном массиве чисел.

\img{120mm}{example}{Пример работы программы}
\clearpage

\section{Время выполнения алгоритмов}

Как было сказано выше, используется функция замера процессорного времени process\_time(...) из библиотеки time на Python. Функция возвращает пользовательское процессорное временя типа float.

Использовать функцию приходится дважды, затем из конечного времени нужно вычесть начальное, чтобы получить результат.

Результаты замеров приведены в таблицах \ref{tbl:best} - \ref{tbl:random} (время в мс).

\begin{table}[h]
	\begin{center}
		\begin{threeparttable}
		\captionsetup{singlelinecheck=off}
		\caption{Отсортированные данные}
		\label{tbl:best}
		\begin{tabular}{|c|c|c|c|}
			\hline
			Размер & Расческой & Бинарным деревом & Слиянием \\
			\hline
			  100 & 96.68 & 555.99 & 95.46 \\ 
			\hline
			200 & 220.52 & 2263.25 & 222.92 \\ 
			\hline
			300 & 417.48 & 5007.74 & 349.33 \\ 
			\hline
			400 & 577.89 & 8962.32 & 471.86 \\ 
			\hline
			500 & 794.07 & 14073.85 & 604.17 \\ 
			\hline
			600 & 984.98 & 20987.65 & 802.39 \\ 
			\hline
			700 & 1286.17 & 29853.94 & 938.30 \\ 
			\hline
			800 & 1486.94 & 38480.89 & 1073.84 \\ 
			\hline
			900 & 1774.24 & 49576.84 & 1199.23 \\ 
			\hline
		\end{tabular}
		\end{threeparttable}
    \end{center}
\end{table}

\begin{table}[h]
	\begin{center}
		\begin{threeparttable}
		\captionsetup{singlelinecheck=off}
		\caption{Отсортированные в обратном порядке данные}
		\label{tbl:worth}
		\begin{tabular}{|c|c|c|c|}
			\hline
			 Размер & Расческой & Бинарным деревом & Слиянием \\
			\hline
			  100 & 102.80 & 557.37 & 105.87 \\ 
			\hline
			200 & 231.22 & 2224.29 & 224.09 \\ 
			\hline
			300 & 445.14 & 5045.33 & 352.55 \\ 
			\hline
			400 & 601.85 & 9001.34 & 463.68 \\ 
			\hline
			500 & 810.16 & 14074.22 & 606.80 \\ 
			\hline
			600 & 1045.80 & 20323.24 & 764.61 \\ 
			\hline
			700 & 1282.02 & 27838.25 & 907.07 \\ 
			\hline
			800 & 1473.24 & 39628.49 & 1161.71 \\ 
			\hline
			900 & 1888.61 & 50180.55 & 1242.45 \\ 
			\hline
		\end{tabular}
		\end{threeparttable}
    \end{center}
\end{table}

\begin{table}[h]
	\begin{center}
		\begin{threeparttable}
		\captionsetup{singlelinecheck=off}
		\caption{Случайные данные}
		\label{tbl:random}
		\begin{tabular}{|c|c|c|c|}
			\hline
			 Размер & Расческой & Бинарным деревом & Слиянием \\
			\hline
			  100 & 108.12 & 99.93 & 107.90 \\ 
			\hline
			200 & 255.87 & 253.64 & 249.29 \\ 
			\hline
			300 & 470.66 & 400.53 & 400.66 \\ 
			\hline
			400 & 678.91 & 585.54 & 539.63 \\ 
			\hline
			500 & 922.55 & 763.61 & 690.55 \\ 
			\hline
			600 & 1110.81 & 935.66 & 856.70 \\ 
			\hline
			700 & 1370.43 & 1202.98 & 1016.88 \\ 
			\hline
			800 & 1576.80 & 1398.82 & 1181.66 \\ 
			\hline
			900 & 1864.26 & 1618.94 & 1371.91 \\ 
			\hline
		\end{tabular}
		\end{threeparttable}
    \end{center}
\end{table}

Также на рисунках \ref{img:sort} - \ref{img:rand} приведены графические результаты замеров работы реализации сортировок в зависимости от размера входного массива.


\img{100mm}{sort}{Отсортированный массив}
\img{100mm}{sort_back}{Отсортированный в обратном порядке массив}
\img{100mm}{rand}{Случайный массив}
\clearpage

\section*{Вывод}
Исходя из полученных результатов, сортировка бинарным деревом при массиве заполненном отсортированными в любом порядке данными работает дольше других примерно в 15 раз, но на случайных значениях она примерно в 1.13 раз медленнее, чем сортировка расческой и в 1.2 раза быстрее сортировки слиянием.
Так же из эксперимента видно, что сортировка слиянием показала себя лучше всех при размере массива более 200 на любых данных, а при отсортированных в прямом и обратном порядках данных она быстрее, чем сортировка расческой примерно в 1.3 раза.
