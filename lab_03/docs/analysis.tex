\chapter{Аналитическая часть}
В этом разделе будут рассмотрены алгоритмы сортировок --- расческой, бинарным деревом и слиянием.

\section{Сортировка расческой}
Сортировка расческой является модификацией сортировки пузырьком, и конкурирует с алгоритмами, подобными быстрой сортировке. Основная идея --- устранить маленькие значения в конце списка, которые крайне замедляют сортировку пузырьком. В сортировке пузырьком, когда сравниваются два элемента, промежуток равен 1. В сортировке расческой этот промежуток начинается с большого значения и уменьшается в 1.3 раза каждую итерацию, пока не станет единицей. В среднем алгоритм работает лучше, чем пузырьковая сортировка. Худший случай, однако остается $O(n^2)$ \cite{comb}.

\section{Сортировка бинарным деревом}

Сортировка бинарным деревом --- универсальный алгоритм сортировки, заключающийся в построении двоичного дерева поиска по ключам массива, с последующей сборкой результирующего массива путём обхода узлов построенного дерева в необходимом порядке следования ключей \cite{Knut}.

Шаги алгоритма:
\begin{enumerate}[label=\arabic*)]
	\item построить двоичное дерево поиска по ключам массива;
	\item собрать результирующий массив путём обхода узлов дерева поиска в необходимом порядке следования ключей;
	\item вернуть, в качестве результата, отсортированный массив.
\end{enumerate}

\section{Сортировка слиянием}
Данная сортировка была разработана Джоном фон Нейманом в 1945 году и относится к классу рекурсивных алгоритмов \cite{algos}. 

Алгоритм работает следующим образом.
\begin{enumerate}[label=\arabic*)]
	\item Массив рекурсивно разбивается на 2 равные части, и каждая из частей делится до тех пор, пока размер очередного подмассива не станет равным единице.	
	\item Далее выполняется операция алгоритма, называемая слиянием. Два единичных массива сливаются в общий результирующий массив, при этом из каждого выбирается меньший элемент (при сортировке по возрастанию) и записывается в свободную левую ячейку результирующего массива. После чего из двух результирующих массивов собирается третий общий отсортированный массив, и так далее. В случае, если один из массивов закончится, элементы другого дописываются в собираемый массив.	
	\item В конце операции слияния, элементы перезаписываются из результирующего массива в исходный.
\end{enumerate}


\section*{Вывод}
В данном разделе были описаны идеи рассматриваемых алгоритмов сортировки: расческой, бинарным деревом и слиянием.