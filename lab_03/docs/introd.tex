\chapter*{Введение}
\addcontentsline{toc}{chapter}{Введение}

Сортировка --- одна из основных операций в алгоритмике и программировании, которая позволяет упорядочить данные по заданному критерию. Однако, сортировка может быть времязатратной операцией, особенно при работе с большими объемами данных. Поэтому важно понимать и анализировать трудоемкость различных алгоритмов сортировки.

В данной лабораторной работе мы рассмотрим несколько известных алгоритмов сортировки, таких как сортировка расческой, сортировка бинарным деревом и сортировка слиянием. Будет проведена теоретическая оценка их трудоемкости. Для подтверждения результатов будет выполнен экспериментальный анализ, в ходе которого мы проанализируем время выполнения каждого алгоритма на разных наборах данных.


\textbf{Целью данной работы} является изучение и исследование трудоемкости алгоритмов сортировки - сортировка расческой, сортировка бинарным деревом и сортировка слиянием.
Для достижения поставленной цели необходимо выполнить следующие задачи:
\begin{enumerate}
	\item описать и реализовать алгоритмы сортировки слиянием, бинарным деревом и расческой;
    \item провести экспиремент, чтобы измерить время выполнения для выбранных сортировок;
    \item провести сравнительный анализ трудоемкости алгоритмов на основе теоретических расчетов;
    \item провести сравнительный анализ процессорного времени реализаций алгоритмов;
	\item описать и обосновать полученные результаты в отчете о выполненной лабораторной работе.
\end{enumerate}