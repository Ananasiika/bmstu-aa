\chapter{Технологическая часть}

В данном разделе будут рассмотрены средства реализации, а также представлены листинги сортировок.

\section{Средства реализации}
В данной работе для реализации был выбран язык программирования $Python \cite{python-lang}$. В текущей лабораторной работе требуется замерить процессорное время для выполняемой программы, а также построить графики. Все эти инструменты присутствуют в выбранном языке программирования.

Время работы было замерено с помощью функции \textit{process\_time(...)} из библиотеки $time \cite{python-lang-time}$.

\section{Описание используемых типов данных}

При реализации алгоритмов будут использованы следующие типы данных:

\begin{itemize}
	\item количество элементов в массиве --- целое число типа \textit{int};
	\item массив --- список типа \textit{int}.
\end{itemize}

\section{Сведения о модулях программы}
Программа состоит из двух модулей:
\begin{itemize}
	\item $main.py$ --- файл, содержащий весь служебный код;
    \item $sorts.py$ --- файл, содержащий код всех сортировок. \newline
\end{itemize}


\section{Реализация алгоритмов}

В листингах \ref{lst:bin_tree_sort}-\ref{lst:merge_sort} представлены реализации алгоритмов сортировок (расческой, бинарным деревом и слиянием).

\begin{center}
\captionsetup{justification=raggedright,singlelinecheck=off}
	\begin{lstlisting}[label=lst:bin_tree_sort,caption=Алгоритм сортировки бинарным деревом]
class Node:
	def __init__(self, value):
		self.value = value
		self.left = None
		self.right = None
def insert_node(root, value):
	if root is None:
		return Node(value)
	if value < root.value:
		root.left = insert_node(root.left, value)
	else:
		root.right = insert_node(root.right, value)
	return root
def centered_traversal(root, sorted_list):
	if root:
		centered_traversal(root.left, sorted_list)
		sorted_list.append(root.value)
		centered_traversal(root.right, sorted_list)
def binary_tree_sort(arr):
	root = None
	for element in arr:
		root = insert_node(root, element)
	sorted_list = []
	centered_traversal(root, sorted_list)
	return sorted_list
	\end{lstlisting}
\end{center}
\clearpage
\begin{center}
	\captionsetup{justification=raggedright,singlelinecheck=off}
	\begin{lstlisting}[label=lst:comb_sort,caption=Алгоритм сортировки расческой]
def comb_sort(arr):
	gap = len(arr)
	shrink_factor = 1.3
	while gap >= 1:
		gap = int(gap / shrink_factor)
		i = 0
		while i + gap < len(arr):
			if arr[i] > arr[i + gap]:
				arr[i], arr[i + gap] = arr[i + gap], arr[i]
			i += 1
	return arr
	\end{lstlisting}
\end{center}

\begin{center}
\captionsetup{justification=raggedright,singlelinecheck=off}
	\begin{lstlisting}[label=lst:merge_sort,caption=Алгоритм сортировки слиянием]
def merge_sort(arr):
	if len(arr) <= 1:
		return arr
	mid = len(arr) // 2
	left = merge_sort(arr[:mid])
	right = merge_sort(arr[mid:])
	return merge(left, right)
def merge(left, right):
	merged = []
	i = j = 0
	while i < len(left) and j < len(right):
		if left[i] < right[j]:
			merged.append(left[i])
			i += 1
		else:
			merged.append(right[j])
			j += 1
	while i < len(left):
		merged.append(left[i])
		i += 1
	while j < len(right):
		merged.append(right[j])
		j += 1
	return merged
	\end{lstlisting}
\end{center}

\section{Функциональные тесты}

В таблице \ref{tbl:functional_test} приведены тесты для функций, реализующих алгоритмы сортировки. Тесты \textit{для всех сортировок} пройдены успешно.


\begin{table}[h]
	\begin{center}
		\begin{threeparttable}
		\captionsetup{singlelinecheck=off, justification=raggedright}
		\caption{\label{tbl:functional_test} Функциональные тесты}
		\begin{tabular}{|c|c|c|}
			\hline
			Входной массив & Ожидаемый результат & Результат \\ 
			\hline
			$[1, 2, 3, 4, 5]$ & $[1, 2, 3, 4, 5]$  & $[1, 2, 3, 4, 5]$\\
			\hline
			$[5, 4, 3, 2, 1]$  & $[1, 2, 3, 4, 5]$ & $[1, 2, 3, 4, 5]$\\
			\hline
			$[9, 7, -5, 1, 4]$  & $[-5, 1, 4, 7, 9]$  & $[-5, 1, 4, 7, 9]$\\
			\hline
			$[5]$  & $[5]$  & $[5]$\\
			\hline
			$[]$  & $[]$  & $[]$\\
			\hline
		\end{tabular}
    \end{threeparttable}
	\end{center}
\end{table}


\section*{Вывод}

Были представлены средства реализации и сами реализации каждого алгоритма.
