\chapter*{Заключение}
\addcontentsline{toc}{chapter}{Заключение}

В результате вычисления трудоемкости алгоритмов сортировок было определено, что алгоритм сортировки бинарным деревом и алгоритм сортировки слиянием и в наилучшем, и в наихудшем случаях имеют трудоемкость, равную $Nlog(N)$, а сортировка расческой в лучшем случае имеет трудоемкость, равную тоже $Nlog(N)$, а в худшем случае $N^2$.

В результате эксперимента было получено, что сортировка бинарным деревом при массиве заполненном отсортированными в любом порядке данными работает дольше других примерно в 15 раз, но на случайных значениях она примерно в 1.13 раз медленнее, чем сортировка расческой и в 1.2 раза быстрее сортировки слиянием.

Так же из эксперимента видно, что сортировка слиянием показала себя лучше всех при размере массива более 200 на любых данных, а при отсортированных в прямом и обратном порядках данных она быстрее, чем сортировка расческой примерно в 1.3 раза.

В ходе лабораторной работы были описаны и реализованы алгоритмы сортировки слиянием, бинарным деревом и расческой, проведен экспиремент, чтобы измерить время выполнения для этих сортировок. Также был проведен сравнительный анализ трудоемкости алгоритмов на основе теоретических расчетов, анализ процессорного времени реализаций алгоритмов и составлен отчет с описанием и обоснованием полученных результатов.

Все задачи лабораторной работы были выполнены и, следовательно, поставленная цель была достигнута.