\chapter{Аналитическая часть}
В этой части будут даны определения составляемых графовых моделей алгоритмов.

\section{Графовые модели программы}

Программа представлена в виде графа: набор вершин и множество соединяющих их направленных дуг.
Дуги отражают связь (отношение) между вершинами~\cite{graph}.

Выделяют 2 типа дуг:
\begin{enumerate}[label=\arabic*)]
	\item операционное отношение --- две вершины A и B соединяются направленной дугой тогда и только тогда, когда вершина B может быть выполнена сразу после вершины A;
	\item информационное отношение --- две вершины A и B соединяются направленной дугой тогда и только тогда, когда вершина использует в качестве аргумента некоторое значение, полученное в вершине A.
\end{enumerate}

Граф управления --- модель, в которой вершины --- операторы, дуги --- операционные отношения.

Информационный граф --- модель, в которой вершины --- операторы, дуги --- информационные отношения.

Операционная история --- модель, в которой вершины --- срабатывание операторов, дуги --- операционные отношения.

Информационная история --- модель, в которой вершины --- срабатывание операторов, дуги --- информационные отношения.

Графы более компактны, а истории более информативны. 
