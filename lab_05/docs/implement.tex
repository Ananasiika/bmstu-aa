\chapter{Технологическая часть}

В данном разделе будут рассмотрены средства реализации, а также представлены листинги линейной и конвейерной орбаботок текста.

\section{Средства реализации}
В данной работе для реализации был выбран язык программирования $C++$~\cite{cpp-lang}. 
В текущей лабораторной работе требуется замерить процессорное время для выполняемой программы, а также реализовать принципы многопоточного алгоритма. 
Все эти инструменты присутствуют в выбранном языке программирования. 

Функции построения графиков были реализованы с использованием языка программирования $Python$~\cite{python-lang}.

Время замерено с помощью \textit{std::chrono::system\_clock::now(...)} --- функции из библиотеки $chrono$~\cite{cpp-lang-chrono}.


\section{Реализация алгоритмов}

В листинге \ref{lst:linear_alg} представлена реализация алгоритма линейной обработки текста, а в листингах \ref{lst:par_alg}--\ref{lst:3} --- реализация алгоритма конвейерной обработки текста.

\clearpage

\begin{center}
    \captionsetup{justification=raggedright,singlelinecheck=off}
    \begin{lstlisting}[label=lst:linear_alg,caption=Алгоритм линейной обработки текста]
void parse_linear(int count, std::vector<std::string> filenames, bool is_print)
{
	time_now = 0;
	std::queue<std::string> q1;
	std::queue<std::string> q2;
	std::queue<std::vector<std::string>> q3;
	queues_t queues = {.q1 = q1, .q2 = q2, .q3 = q3};
	for (int i = 0; i < count; i++)
	{
		queues.q1.push(filenames[i]);
		std::string filename = queues.q1.front();
		stage1_linear(filename, i + 1, is_print);
		queues.q1.pop();
		queues.q2.push(filename);
		std::string text = queues.q2.front();
		stage2_linear(filename, i + 1, is_print); // Stage 2
		queues.q2.pop();
		queues.q3.push(filename);
		std::vector<std::string> sentences = queues.q3.front();
		stage3_linear(filename, i + 1, is_print); // Stage 3
		queues.q3.pop();
	}
}
\end{lstlisting}
\end{center}

\clearpage

\begin{center}
    \captionsetup{justification=raggedright,singlelinecheck=off}
    \begin{lstlisting}[label=lst:par_alg,caption=Алгоритм конвейерной обработки текста]
void parse_parallel(int count, std::vector<std::string> filenames, bool is_print)
{
	time_now = 0;
	std::queue<matrix_t> q1;
	std::queue<matrix_t> q2;
	std::queue<matrix_t> q3;
	queues_t queues = {.q1 = q1, .q2 = q2, .q3 = q3};
	for (int i = 0; i < count; i++)
	{
		q1.push(filenames[i]);
	}
	std::thread threads[THREADS];
	threads[0] = std::thread(stage1_parallel, std::ref(q1), std::ref(q2), std::ref(q3), is_print);
	threads[1] = std::thread(stage2_parallel, std::ref(q1), std::ref(q2), std::ref(q3), is_print);
	threads[2] = std::thread(stage3_parallel, std::ref(q1), std::ref(q2), std::ref(q3), is_print);
	for (int i = 0; i < THREADS; i++)
	{
		threads[i].join();
	}
}
\end{lstlisting}
\end{center}


\clearpage


\begin{center}
    \captionsetup{justification=raggedright,singlelinecheck=off}
    \begin{lstlisting}[label=lst:1,caption=Алгоритм запуска 1 потока для чтения текста из файла]
void stage1_parallel(std::queue<std::string> &q1, std::queue<std__string> &q2, std::queue<std::vector<std::string>> &q3, bool is_print)
{
	int task_num = 1;
	std::mutex m;
	while(!q1.empty())
	{      
		m.lock();
		std::string filename = q1.front();
		m.unlock();
		log(filename, task_num++, 1, read_file, is_print);
		m.lock();
		q2.push(filename);
		q1.pop();
		m.unlock();
	}
}
\end{lstlisting}
\end{center}
\begin{center}
	\captionsetup{justification=raggedright,singlelinecheck=off}
	\begin{lstlisting}[label=lst:3,caption=Алгоритм запуска 3 потока для построения деревьев синтаксических зависимостей]
void stage3_parallel(std::queue<std::string> &q1, std::queue<std::string> &q2, std::queue<std::vector<std::string>> &q3, bool is_print) {
	int task_num = 1;
	std::mutex m;
	do { 
		m.lock();
		bool is_q3empty = q3.empty();
		m.unlock();
		if (!is_q3empty) {
			m.lock();
			std::vector<std::string> sentences = q3.front(); 
			q3.pop();
			m.unlock();
			log(filename, task_num++, 3, tree, is_print);
		}
	} while (!q1.empty() || !q2.empty() || !q3.empty());
}
	\end{lstlisting}
\end{center}

\clearpage

\begin{center}
    \captionsetup{justification=raggedright,singlelinecheck=off}
    \begin{lstlisting}[label=lst:2,caption=Алгоритм запуска 2 потока для разделения текста на предложения]
void stage2_parallel(std::queue<std::string> &q1, std::queue<std::string> &q2, std::queue<std::vector<std::string>> &q3, bool is_print)
{
	int task_num = 1;
	std::mutex m;
	do
	{   
		m.lock();
		bool is_q2empty = q2.empty();
		m.unlock();
		if (!is_q2empty)
		{   
			m.lock();
			std::string filename = q2.front();
			m.unlock();
			log(filename, task_num++, 2, get_sentences, is_print);
			m.lock();
			q3.push(filename);
			q2.pop();
			m.unlock();
		}
	} while (!q1.empty() || !q2.empty());
}
\end{lstlisting}
\end{center}

\section{Сведения о модулях программы}
Программа состоит из следующих модулей:
\begin{itemize}
	\item \textit{main.cpp} --- файл, содержащий меню программы;
	\item \textit{conveyor.h} и \textit{conveyor.cpp} --- файлы, содержащие код реализации линейной и конвейерной обработок, а также функции замера времени;
	\item \textit{graph\_build.py} --- файл, содержащий функции построения графиков для замеров по времени.
\end{itemize}


\section{Вывод}

Были представлены листинги всех алгоритмов линейной и конвейерной обработки текста. Также в данном разделе была приведена информация о выбранных средствах для разработки алгоритмов и сведения о модулях программы.
