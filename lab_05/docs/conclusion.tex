\chapter*{Заключение}
\addcontentsline{toc}{chapter}{Заключение}

Результаты эксперимента показали, что при конвейерной обработке время выполнения показывает лучшие результаты по сравнению с линейной реализацией. 
При количестве файлов, равном 4, время выполнения конвейерной реализации оказалось в 1.2 раза меньше, а при количестве файлов, равном 9, --- в 1.3 раза меньше. 
Это свидетельствует о преимуществе конвейерной реализации при увеличении количества задач (файлов).

Также при увеличении размера файла обнаружено, что конвейерная реализация продемонстрировала более высокую производительность. 
Например, при размере файла 100 кб время выполнения конвейерной реализации оказалось в 1.2 раза меньше, чем линейной реализации, а при размере файла 900 кб --- уже в 1.3 раза меньше.

Цель, которая была поставлена в начале лабораторной работы была достигнута, а также в ходе выполнения лабораторной работы были решены следующие задачи:

\begin{itemize}[label=---]
	\item описаны основы конвейерной обработки данных и алгоритма построения дерева синтаксических зависимостей;
	\item приведены схемы конвейерной и линейной обработок;
	\item реализованы разработанные алгоритмы;
	\item проведен сравнительный анализ по времени для реализованных алгоритмов;
	\item подготовлен отчет о выполненной лабораторной работе.
\end{itemize}
