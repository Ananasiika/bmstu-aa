\chapter*{Введение}
\addcontentsline{toc}{chapter}{Введение}

Конвейерная обработка является одним из примеров, где использование принципов параллельности помогает ускорить обработку данных. 
Суть та же, что и при работе реальных конвейерных лент --- материал поступает на обработку, после окончания обработки материал передается на место следующего обработчика, при этом предыдыдущий обработчик не ждет полного цикла обработки материала, а получает новый материал и работает с ним.


\textbf{Целью данной работы} является исследование принципов конвейерной обработки данных. 
Для достижения поставленной цели необходимо выполнить следующие задачи:
\begin{itemize}[label=---]
	\item описать основы конвейерной обработки данных и алгоритма построения дерева синтаксических зависимостей, которые будут использоваться в текущей лабораторной работе;
    \item привести схемы конвейерной и линейной обработок;
    \item реализовать разработанные алгоритмы;
    \item провести сравнительный анализ по времени для реализованных алгоритмов;
    \item подготовить отчет о выполненной лабораторной работе.
\end{itemize}
