\chapter{Аналитическая часть}
В этом разделе будет представлена информация по поводу сути конвейерной обработки данных.


\section{Конвейерная обработка данных}

\textbf{Конвейер} \cite{conveyer} -- организация вычислений, при которой увеличивается количество выполняемых инструкций за единицу времени за счет использования принципов параллельности.

Конвейеризация в общем случае основана на разделении подлежащей исполнению функции на более мелкие части, называемые ступенями, и выделении для каждой из них отдельного блока аппаратуры. 
Так, обработку любой машинной команды можно разделить на несколько этапов (несколько ступеней), организовав передачу данных от одного этапа к следующему. 
При этом конвейерную обработку можно использовать для совмещения этапов выполнения разных команд.
Производительность при этом возрастает, благодаря тому, что одновременно на различных ступенях конвейера выполняется несколько команд. 
Конвейерная обработка такого рода широко применяется во всех современных быстродействующих процессорах.

Конвейеризация увеличивает пропускную способность процессора (количество команд, завершающихся в единицу времени), но она не сокращает время выполнения отдельной команды. 
В действительности она даже несколько увеличивает время выполнения каждой команды из-за накладных расходов, связанных с хранением промежуточных результатов. 
Однако увеличение пропускной способности означает, что программа будет выполняться быстрее по сравнению с простой, неконвейерной схемой.

\section[Деревья синтаксических зависимостей текста]{Деревья синтаксических зависимостей\\ текста}

Дерево синтаксических зависимостей представляет собой структуру, которая отображает отношения между словами в предложении или тексте. 
Оно является графическим представлением синтаксической структуры предложения и позволяет анализировать связи между словами.

В дереве синтаксических зависимостей каждое слово представлено узлом, а связи между словами представлены направленными ребрами. 
Узлы могут иметь различные свойства, такие как лемма (нормальная форма слова), часть речи, морфологические характеристики и другие.

Каждое ребро в дереве синтаксических зависимостей указывает на отношение между словами. 
Например, ребро может указывать на то, что слово является зависимым от другого слова в качестве подлежащего, сказуемого, прямого или косвенного дополнения и т.д. 
Все эти отношения образуют иерархическую структуру, которая отображает семантику и синтаксис в предложении.

Построение дерева синтаксических зависимостей основано на лексическом и синтаксическом анализе текста. 
При анализе предложения или текста, алгоритм разбивает его на токены (слова), определяет их часть речи, а затем выстраивает связи между словами, чтобы создать дерево синтаксических зависимостей.

\section{Описание алгоритмов}

В качестве примера для конвейерной обработки будет обрабатываться текст. Всего будет использовано три ленты, которые делают следующее.

\begin{enumerate}[label=\arabic*)]
	\item Чтение текста из файла.
	\item Разбиение текста на предложения.
	\item Построение деревьев синтаксических зависимостей для предложений.
\end{enumerate}


\section{Вывод}

В данном разделе было рассмотрено понятие конвейерной обработки, а также выбраны этапы для обработки текста, которые будут обрабатывать ленты конвейера.