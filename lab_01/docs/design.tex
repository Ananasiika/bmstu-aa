\chapter{Конструкторская часть}
В этом разделе будет представлено описание используемых типов данных и схемы алгоритмов вычисления расстояния Левенштейна и Дамерау-Левенштейна.


\section{Сведения о модулях программы}
Программа состоит из двух модулей:
\begin{itemize}
	\item $main.py$ --- основной файл программы, в котором находится главный код, выполняющийся при запуске программы;
    \item $algorythms.py$ --- файл, содержащий код всех алгоритмов. \newline
\end{itemize}


\section{Разработка алгоритмов}
На рисунках \ref{img:lev}-\ref{img:dam_lev_cache} представлены схемы алгоритмов вычисления расстояния Левенштейна и Дамерау-Левенштейна.

\imgScale{0.53}{lev}{Схема матричного алгоритма нахождения расстояния Левенштейна}
\imgScale{0.49}{dam_lev}{Схема матричного алгоритма нахождения расстояния Дамерау-Левенштейна}
\imgScale{0.6}{dam_lev_rec}{Схема рекурсивного алгоритма нахождения расстояния Дамерау-Левенштейна}
\imgScale{0.57}{dam_lev_cache}{Схема рекурсивного алгоритма нахождения расстояния Дамерау-Левенштейна с использованием кеша}

\clearpage

\section{Классы эквивалентности тестирования}

Для тестирования выделены классы эквивалентности, представленные ниже.

\begin{enumerate}[label=\arabic*)]
    \item Две пустые строки.
    \item Одна из строк пустая.
    \item Расстояния, которые вычислены алгоритмами Левенштейна и Дамерау-Левенштейна, равны.
    \item Расстояния, которые вычислены алгоритмами Левенштейна и Дамерау-Левенштейна, не равны.
\end{enumerate}


\section{Вывод}
В данном разделе было представлено описание используемых типов данных и схемы алгоритмов, рассматриваемых в лабораторной работе.