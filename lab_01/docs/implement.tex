\chapter{Технологическая часть}

В данном разделе будут рассмотрены средства реализации, а также представлены листинги алгоритмов определения расстояния Левенштейна и Дамерау-Левенштейна.

\section{Средства реализации}
В данной работе для реализации был выбран язык программирования $Python \cite{python-lang}$. В текущей лабораторной работе требуется замерить процессорное время для выполняемой программы, а также построить графики. Все эти инструменты присутствуют в выбранном языке программирования.

Время работы было замерено с помощью функции \textit{process\_time()} из библиотеки $time \cite{python-lang-time}$.


\section{Описание используемых типов данных}
При реализации алгоритмов будут использованы следующие структуры данных:

\begin{itemize}
	\item две строки типа \textit{str};
	\item длина строки --- целое число типа \textit{int};
	\item в матричных реализациях алгоритмов Левенштейна и Дамерау-Левен-\\штейна дополнительно используется матрица, которая является двумерным списком типа \textit{int}. 
\end{itemize}


\section{Реализация алгоритмов}

В листингах \ref{lst:lev_rec}-\ref{lst:dam_lev_rec} представлены реализации алгоритмов нахождения расстояния Левенштейна и Дамерау-Левенштейна.
\begin{center}
    \captionsetup{justification=raggedright,singlelinecheck=off}
    \begin{lstlisting}[label=lst:lev_rec,caption=Алгоритм нахождения расстояния Левенштейна (матричный)]
def levenshtein_matrix(str1, str2):
	m = len(str1)
	n = len(str2)
	matrix = [[0] * (m + 1) for _ in range(n + 1)]
	for i in range(n + 1):
		matrix[i][0] = i
	for j in range(m + 1):
		matrix[0][j] = j
	for i in range(1, m + 1):
		for j in range(1, n + 1):
			cost = 0 if str1[i - 1] == str2[j - 1] else 1
			matrix[i][j] = min(matrix[i - 1][j] + 1,         
			matrix[i][j - 1] + 1,         
			matrix[i - 1][j - 1] + cost)  
	return matrix[m][n]
\end{lstlisting}
\end{center}


\begin{center}
    \captionsetup{justification=raggedright,singlelinecheck=off}
    \begin{lstlisting}[label=lst:lev_mat,caption=Алгоритм нахождения расстояния Дамерау-Левенштейна]
def damerau_levenshtein_matrix(str1, str2):
	m = len(str1)
	n = len(str2)
	matrix = [[0] * (m + 1) for _ in range(n + 1)]
	for i in range(n + 1):
		matrix[i][0] = i
	for j in range(m + 1):
		matrix[0][j] = j
	for i in range(1, m + 1):
		for j in range(1, n + 1):
			cost = 0 if str1[i - 1] == str2[j - 1] else 1
			matrix[i][j] = min(matrix[i - 1][j] + 1,         
			matrix[i][j - 1] + 1,        
			matrix[i - 1][j - 1] + cost) 
	if i > 1 and j > 1 and str1[i - 1] == str2[j - 2] and str1[i - 2] == str2[j - 1]:
		matrix[i][j] = min(matrix[i][j], matrix[i - 2][j - 2] + 1)
	return matrix[m][n]
\end{lstlisting}
\clearpage
    \captionsetup{justification=raggedright,singlelinecheck=off}
    \begin{lstlisting}[label=lst:lev_cach,caption=Рекурсивный алгоритм нахождения расстояния Дамерау-Левенштейна]
def damerau_levenshtein_recursive(str1, str2):
	if len(str1) == 0:
		return len(str2)
	if len(str2) == 0:
		return len(str1)
	flag = 0 if (str1[-1] == str2[-1]) else 1
	add = damerau_levenshtein_recursive(str1[:-1], str2) + 1
	delete = damerau_levenshtein_recursive(str1, str2[:-1]) + 1
	change = damerau_levenshtein_recursive(str1[:-1], str2[:-1]) + flag
	transposition = damerau_levenshtein_recursive(str1[:-2], str2[:-2]) + 1
	if (len(str1) > 1 and len(str2) > 1 and str1[-1] == str2[-2] and str1[-2] == str2[-1]):
		minimum  = min(add, delete, change, transposition)
	else:
		minimum = min(add, delete, change)
	return minimum
\end{lstlisting}
\end{center}
\clearpage

\begin{center}
    \captionsetup{justification=raggedright,singlelinecheck=off}
    \begin{lstlisting}[label=lst:dam_lev_rec,caption=Рекурсивный алгоритм нахождения расстояния Дамерау-Левенштейна с использованием кеша]
def damerau_levenshtein_cache_recursive(str1, str2):
	cache = {}
	def damerau_levenshtein_recursive(str1, str2):
		if (str1, str2) in cache:
			return cache[(str1, str2)]
		if len(str1) == 0:
			return len(str2)
		if len(str2) == 0:
			return len(str1)
		deletion = damerau_levenshtein_recursive(str1[:-1], str2) + 1
		insertion = damerau_levenshtein_recursive(str1, str2[:-1]) + 1
		substitution = damerau_levenshtein_recursive(str1[:-1], str2[:-1]) + (str1[-1] != str2[-1])
		transposition = float('inf')
		if len(str1) > 1 and len(str2) > 1 and str1[-1] == str2[-2] and str1[-2] == str2[-1]:
			transposition = damerau_levenshtein_recursive(str1[:-2], str2[:-2]) + 1
		result = min(deletion, insertion, substitution, transposition)
		cache[(str1, str2)] = result
		return result
	distance = damerau_levenshtein_recursive(str1, str2)
	return distance
\end{lstlisting}
\end{center}


\section{Классы эквивалентности тестирования}

Для тестирования выделены классы эквивалентности, представленные ниже.

\begin{enumerate}[label=\arabic*)]
	\item Две пустые строки.
	\item Одна из строк пустая.
	\item Расстояния, которые вычислены алгоритмами Левенштейна и Дамерау-Левенштейна, равны.
	\item Расстояния, которые вычислены алгоритмами Левенштейна и Дамерау-Левенштейна, не равны.
\end{enumerate}

\section{Функциональные тесты}

В таблице \ref{tbl:functional_test} приведены тесты для функций, реализующих алгоритмы нахождения расстояния Левенштейна и Дамерау-Левенштейна. Тесты \textit{для всех алгоритмов} пройдены успешно.

\begin{table}[h]
	\begin{center}
        \begin{threeparttable}
        \captionsetup{justification=raggedright,singlelinecheck=off}
		\caption{\label{tbl:functional_test} Функциональные тесты}
		\begin{tabular}{|c|c|c|c|c|}
			\hline
			& & & \multicolumn{2}{c|}{Ожидаемый результат} \\
			\hline
			№&Строка 1&Строка 2&Левенштейн&Дамерау-Л. \\
			\hline
            1&""&""&0&0 \\
            \hline
            2&""&слово&5&5 \\
            \hline
            3&слова&""&5&5 \\
            \hline
            4&кот&ком&1&1 \\
			\hline
			5&парк&прак&2&1 \\
			\hline
            6&машина&сирена&4&4 \\
			\hline
			7&здравствуйте&до свидания&10&10 \\
			\hline
            8&кошка&собака&3&3 \\
			\hline
            9&карта&карат&2&1 \\
			\hline
            10&спать&встать&2&2 \\
			\hline
			11&пока&привет&5&5 \\
			\hline
            12&abcdef&bacfe&4&3 \\
			\hline
		\end{tabular}
        \end{threeparttable}
	\end{center}
\end{table}

\section{Вывод}

Были представлены алгоритмы нахождения расстояния Левенштейна и Дамерау-Левенштейна, которые были описаны в предыдущем разделе. Также была приведена информация о выбранных средствах для разработки алгоритмов.
