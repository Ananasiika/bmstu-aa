\begin{center}
	ЗАКЛЮЧЕНИЕ
\end{center}
\addcontentsline{toc}{chapter}{ЗАКЛЮЧЕНИЕ}

В рамках данной лабораторной работы были изучены, реализованы и протестированы алгоритмы нахождения расстояния Левенштейна и Дамерау-Левенштейна. Также был проведён сравнительный анализ алгоритмов по затраченному процессорному времени и памяти и подготовлен отчет о лабораторной работе.

Экспериментально было определено, что наименее затратным по времени оказался итеративный алгоритм нахождения расстояния Левенштейна, а наиболее затратным --- рекурсивный алгоритм Дамерау-Левенштейна. Исходя из замеров по памяти, итеративные алгоритмы проигрывают рекурсивным, потому что максимальный размер памяти в них растет, как произведение длин строк, а в рекурсивных --- как сумма длин строк.
Также при проведении эксперимента было выявлено, что на длине строк в 4 символа рекурсивная реализация алгоритма Дамерау-Левенштейна уже в 10 раз медленнее матричной реализации. При увеличении длины строк в геометрической прогрессии растет и время работы рекурсивной реализации. Следовательно, стоит использовать матричную реализацию для строк длиной более 4 символов.

Цель, которая была поставлена в начале лабораторной работы, была достигнута.