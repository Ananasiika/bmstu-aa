\begin{center}
	ВВЕДЕНИЕ
\end{center}
\addcontentsline{toc}{chapter}{ВВЕДЕНИЕ}

Лабораторная работа по поиску редакционных расстояний Левенштейна и Дамерау-Левенштейна посвящена изучению и применению двух популярных метрик для измерения схожести или различия между строками. Эти метрики, названные в честь их авторов Владимира Левенштейна и Дмитрия Дамерау, широко используются в области редактирования текста, биоинформатики, автоматического исправления ошибок и других приложений, где важно оценить разницу между двумя строками.

В этой лабораторной работе будут изучены и реализованы алгоритмы Левенштейна и Дамерау-Левенштейна для вычисления редакционного расстояния между строками. Редакционное расстояние представляет собой минимальное количество операций (вставка, удаление и замена символов), требуемое для преобразования одной строки в другую. Эти операции отражают различные типы изменений, которые могут произойти в строках при их редактировании.

Будет рассмотрен как матричный подход для определения расстояния Левенштейна и Дамерау-Левенштейна с помощью заполнения матрицы, так и рекурсивный подход с использованием кэша для предотвращения повторных вычислений. 

\textbf{Целью данной работы} является изучение, реализация и исследование алгоритмов нахождения расстояний Левенштейна и Дамерау-Левенштей-\\на. 
Для достижения поставленной цели необходимо выполнить следующие задачи:
\begin{itemize}
	\item изучить и реализовать алгоритмы нахождения расстояний Левенштейна и Дамерау-Левенштейна;
    \item провести тестирование, чтобы измерить время выполнения и использование памяти для каждого алгоритма;
    \item провести сравнение процессорного времени выполнения алгоритмов и памяти;
	\item описать и обосновать полученные результаты в отчете о выполненной лабораторной работе.
\end{itemize}
