\begin{center}
	ВВЕДЕНИЕ
\end{center}
\addcontentsline{toc}{chapter}{ВВЕДЕНИЕ}

Редакционное расстояние Левенштейна и Дамерау---Левентшейна представляет собой минимальное количество операций (вставка, удаление и замена символов), требуемое для преобразования одной строки в другую. Эти метрики широко используются в области редактирования текста, биоинформатики, автоматического исправления ошибок и других приложений, где важно оценить разницу между двумя строками.


\textbf{Целью данной работы} является изучение, реализация и исследование алгоритмов нахождения расстояний Левенштейна и Дамерау-Левенштей-\\на. 
Для достижения поставленной цели необходимо выполнить следующие задачи:
\begin{itemize}
	\item изучить и реализовать алгоритмы нахождения расстояний Левенштейна и Дамерау-Левенштейна;
    \item провести тестирование, чтобы измерить время выполнения и использование памяти для каждого алгоритма;
    \item провести сравнение процессорного времени выполнения алгоритмов и памяти;
	\item описать и обосновать полученные результаты в отчете о выполненной лабораторной работе.
\end{itemize}
