\chapter*{Заключение}
\addcontentsline{toc}{chapter}{Заключение}

В результате эксперимента было обнаружено, что при использовании 4 потоков скорость многопоточной реализации алгоритма построения деревьев синтаксических зависимостей в тексте превосходит скорость однопоточной реализации на размере файла, равном 500 кб, примерно в 2.04 раз. Количество использованных потоков соответствует количеству логических ядер на испытуемом ноутбуке (4 ядра). Распределение работы между всеми потоками примерно одинаково, что позволяет достичь лучших результатов на 4 потоках и устранить недоработки, которые возникают при использовании меньшего числа потоков. Это означает, что многопоточная реализация алгоритма представляет собой предпочтительный вариант при обработке таких данных.

Кроме того, в ходе эксперимента было выявлено, что при увеличении размера файла многопоточная реализация демонстрирует еще более заметные преимущества. Например, при размере файла в 300 кб реализация с 4 потоками работает быстрее однопоточной реализации примерно в 2.09 раз, а при размере файла в 1000 кб --- в 2.37 раз.

Также интересным результатом эксперимента стало то, что реализация с использованием 8 дополнительных потоков оказалась эффективнее по времени в сравнении с использованием 4 дополнительных потоков при размере файла более 500 кб. Это может быть связано с некоторыми ограничениями системы и показывает, что оптимальное количество потоков может зависеть от конкретных условий.

В ходе лабораторной работы были выполнены следующие задачи:
\begin{enumerate}[label=\arabic*)]
	\item описаны основы многопоточного программирования;
	\item реализованы алгоритмы построения деревьев синтаксических зависимостей в тексте с использованием многопоточности и без;
	\item проведен сравнительный анализ реализаций рассматриваемого алгоритма по времени при разном размере текста без использования многопоточности и с использованием разного количества потоков;
	\item описаны и обоснованы полученные результаты в отчете о выполненной лабораторной работе.
\end{enumerate} 

%В ходе лабораторной работы были описаны основы многопоточного программирования, реализованы алгоритмы построения дерева синтаксических зависимостей в тексте с использованием многопоточности и без, проведен экспиремент, чтобы измерить время выполнения для этих реализаций. Также был проведен сравнительный анализ времени при разном размере текста и с использованием разного количества потоков и полученные результаты были описаны в отчете. 

Таким образом, все задачи лабораторной работы были выполнены и, следовательно, поставленная цель была достигнута.