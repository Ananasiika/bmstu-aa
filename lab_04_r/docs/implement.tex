\chapter{Технологическая часть}

В данном разделе будут рассмотрены средства реализации, а также представлены листинги алгоритмов построения деревьев синтаксических зависимостей в тексте с распараллеливанием и без него.
\section{Средства реализации}
В данной работе для реализации был выбран язык программирования $C++$~\cite{cpp-lang}. В текущей лабораторной работе требуется замерить процессорное время для выполняемой программы, а также реализовать принципы многопоточного алгоритма. Все эти инструменты присутствуют в выбранном языке программирования. 

Построение деревьев синтаксических зависимостей с помощью библиотеки $spacy$ и графиков для анализа времени были реализованы с использованием языка программирования $Python$~\cite{python-lang}.

Время замерено с помощью \textit{std::chrono::system\_clock::now(...)} --- функции из библиотеки $chrono$~\cite{cpp-lang-chrono}.
\section{Листинги кода}

В листинге~\ref{lst:alg} представлена реализация алгоритма построения деревьев синтаксических зависимостей с чтением исходного текста из файла и сохранением результата в файл. Реализация алгоритма построения деревьев синтаксических зависимостей без многопоточности представлена в листинге \ref{lst:no_par_alg}, а с многопоточностью в приложении.
\clearpage
\begin{center}
	\captionsetup{justification=raggedright,singlelinecheck=off}
	\begin{lstlisting}[label=lst:no_par_alg,caption=Алгоритм построения деревьев синтаксических зависимостей без многопоточности]
void exec_no_parallel(QString size)
{
	std::ifstream inputFile("data/input_" + size.toStdString() + "kb.txt");
	std::string text;
	std::string line;
	while (std::getline(inputFile, line))
		text += line + " ";
	std::vector<std::string> sentences;
	std::string delimiter = ".!?";
	size_t pos = 0;
	std::string token;
	while (true) {
		pos = text.find_first_of(delimiter);
		if (pos == std::string::npos) 
			break;
		token = text.substr(0, pos+1);
		sentences.push_back(token);
		text.erase(0, pos+1);
	}
	if (!text.empty()) 
		sentences.push_back(text);
	std::string filename_prep = "./data/prep/no_" + size.toStdString() + ".txt";
	std::ofstream outputFile(filename_prep);
	if (!outputFile.is_open()) {
		std::cout << "Failed to open output file" <<  std::endl;
		return;
	}	
	for (const auto& sentence : sentences)
		outputFile << sentence << std::endl;
	outputFile.close();
	std::string command = "python3 main.py " + filename_prep + " ./result/no_" + size.toStdString() + "output.txt";
	system(command.c_str());
}
	\end{lstlisting}
\end{center}
\clearpage
\begin{center}
	\captionsetup{justification=raggedright,singlelinecheck=off}
	\begin{lstlisting}[label=lst:alg,caption=Алгоритм построения деревьев синтаксических зависимостей с помощью библиотеки spacy]
import spacy
import sys
nlp = spacy.load("ru_core_news_sm")
with open(sys.argv[1], "r") as file:
	while (1):
		sentence = file.readline()
		if (len(sentence) == 0):
			break
		doc = nlp(sentence)
		dependency_tree = []
		for token in doc:
			dependency_tree.append(token.text + " " + token.dep_ + " " + token.head.text)
		with open(sys.argv[2], "a") as file_output:
			file_output.write("\n".join(dependency_tree))
			file_output.write("\n\n")
	\end{lstlisting}
\end{center}

\section{Сведения о модулях программы}
Программа состоит из следующих модулей:
\begin{enumerate}[label=\arabic*)]
	\item \textit{main.cpp} --- файл, содержащий функцию, вызывающую интерфейс программы;
    \item \textit{mainwindow.h} и \textit{mainwindow.cpp} --- файлы, содержащие код всех методов, реализующих интерфейс программы и взаимодействующие с ним;
	\item \textit{building.h} и \textit{building.cpp} --- файлы, содержащие код реализаций алгоритмов построения деревьев синтаксических зависимостей с многопоточностью и без нее, а также функции замера времени;
	\item \textit{graph.py} --- файл, содержащий функции построения графиков для замеров по времени;
	\item \textit{main.py} --- файл, содержащий само использование библиотеки $spacy$ для построения деревьев синтаксических зависимостей.
\end{enumerate}

\section*{Вывод}

Были представлены листинги всех реализаций алгоритмов. Также в данном разделе была приведена информация о выбранных средствах для разработки и сведения о модулях программы.
