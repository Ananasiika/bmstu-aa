\chapter{Технологическая часть}

В данном разделе будут рассмотрены средства реализации, а также представлены листинги алгоритмов построения дерева синтаксических зависимостей в тексте с распараллеливанием и без него.
\section{Средства реализации}
В данной работе для реализации был выбран язык программирования $C++$~\cite{cpp-lang}. В текущей лабораторной работе требуется замерить процессорное время для выполняемой программы, а также реализовать принципы многопоточного алгоритма. Все эти инструменты присутствуют в выбранном языке программирования. 

Построение дерева синтаксических зависимостей с помощью библиотеки $spacy$ и графиков для анализа времени были реализованы с использованием языка программирования $Python$~\cite{python-lang}.

Время замерено с помощью \textit{std::chrono::system\_clock::now(...)} --- функции из библиотеки $chrono$~\cite{cpp-lang-chrono}.
\section{Листинги кода}

В листинге~\ref{lst:alg} представлена реализация алгоритма построения дерева синтаксических зависимостей с чтением исходного текста из файла и сохранением результата в файл, а в листингах \ref{lst:no_par_alg} и \ref{lst:par_alg} реализация рассматриваемых алгоритмов с использованием многопоточности и без.
\begin{center}
	\captionsetup{justification=raggedright,singlelinecheck=off}
	\begin{lstlisting}[label=lst:no_par_alg,caption=Алгоритм построения дерева синтаксических зависимостей без многопоточности]
void exec_no_parallel(QString size)
{
	QString command = "python3 ./main.py data/input_" + size + "kb.txt ./result/no_" + size + "output.txt";	
	system(command.toLatin1().constData());
}
	\end{lstlisting}
\end{center}
\begin{center}
    \captionsetup{justification=raggedright,singlelinecheck=off}
    \begin{lstlisting}[label=lst:par_alg,caption=Алгоритм построения дерева синтаксических зависимостей с многопоточностью]
void executeCommand(const QString& command) {
	system(command.toLatin1().constData());
}
void exec_parallel(QString size, int threads_count)
{
	QString filename = "data/input_" + size + "kb.txt";
	QFile file(filename);
	if (!file.open(QIODevice::ReadOnly | QIODevice::Text)) {
		qDebug() << "Failed to open input file:" << filename;
		return;
	}
	QTextStream in(&file);
	QString text = in.readAll();
	file.close();
	int partLength = text.length() / threads_count;
	for (int i = 0; i < threads_count; ++i) {
		QString part = text.mid(i * partLength, partLength);
		QString partFile = "data/parts/part_" + size + "kb_" + QString::number(i) + ".txt";
		QFile partFileObj(partFile);
		if(!partFileObj.open(QIODevice::WriteOnly|QIODevice::Text)){
			qDebug() << "Failed to open part file:" << partFile;
			continue;
		}
		QTextStream out(&partFileObj);
		out << part;
		partFileObj.close();
	}
	std::vector<std::thread> threads;
	for (int i = 0; i < threads_count; ++i) {
		QString command = "python3 ./main.py data/parts/part_" + size + "kb_" + QString::number(i) + ".txt ./result/" + size + "output.txt";
		threads.push_back(std::thread(executeCommand, command));
	}
	for (std::thread& t : threads) {
		t.join();
	}
}
\end{lstlisting}
\end{center}

\begin{center}
	\captionsetup{justification=raggedright,singlelinecheck=off}
	\begin{lstlisting}[label=lst:alg,caption=Алгоритм построения дерева синтаксических зависимостей с помощью библиотеки spacy]
parser = argparse.ArgumentParser()
parser.add_argument('input_file')
parser.add_argument('output_file')
args = parser.parse_args()
nlp = spacy.load("ru_core_news_sm")
with open(args.input_file, 'r') as file:
	text = file.read()
doc = nlp(text)
with open(args.output_file, 'a', encoding="utf-8") as file:
	for token in doc:
		file.write(token.text + " " + token.dep_ + " " + token.head.text + '\n')
	\end{lstlisting}
\end{center}

\section{Сведения о модулях программы}
Программа состоит из следующих модулей:
\begin{enumerate}[label=\arabic*)]
	\item \textit{main.cpp} --- файл, содержащий функцию, вызывающую интерфейс программы;
    \item \textit{mainwindow.h} и \textit{mainwindow.cpp} --- файлы, содержащие код всех методов, реализующих интерфейс программы и взаимодействующие с ним;
	\item \textit{building.h} и \textit{building.cpp} --- файлы, содержащие код реализаций алгоритмов построения дерева синтаксических зависимостей с многопоточностью и без нее, а также функции замера времени;
	\item \textit{graph.py} --- файл, содержащий функции построения графиков для замеров по времени;
	\item \textit{main.py} --- файл, содержащий само использование библиотеки $spacy$ для построения дерева синтаксических зависимостей.
\end{enumerate}

\section{Вывод}

Были представлены листинги всех реализаций алгоритмов. Также в данном разделе была приведена информация о выбранных средствах для разработки и сведения о модулях программы.
