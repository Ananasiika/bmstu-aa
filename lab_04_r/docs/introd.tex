\chapter*{Введение}
\addcontentsline{toc}{chapter}{Введение}

Многопоточность является одной из важных концепций в разработке программного обеспечения. Она позволяет выполнять несколько задач параллельно, увеличивая эффективность работы приложений и улучшая отзывчивость пользовательского интерфейса. В современном мире, где все больше задач требует обработки больших объемов данных или выполнения сложных вычислений, эффективное использование многопоточности становится критически важным.

\textbf{Целью данной работы} является изучение параллельных вычислений на основе построения деревьев синтаксических зависимостей текстов при помощи библиотеки синтаксического анализа (spacy).

Для достижения поставленной цели необходимо выполнить следующие задачи:
\begin{enumerate}[label=\arabic*)]
	\item описать основы многопоточного программирования;
    \item реализовать алгоритм построения дерева синтаксических зависимостей в тексте с использованием многопоточности и без;
    \item провести сравнительный анализ по времени при разном размере текста без использования многопоточности и с использованием разного количества потоков;
	\item описать и обосновать полученные результаты в отчете о выполненной лабораторной работе, выполненного как расчётно-пояснительная записка к работе.
\end{enumerate}
