\chapter{Аналитическая часть}
В этом разделе будет представлена информация по поводу многопоточности и деревьев синтаксических зависимостей в тексте, построение которых будет распараллелено в данной лабораторной работе.

\section{Многопоточность}

Многопоточность --- это возможность процессора выполнять несколько потоков одновременно в рамках использования ресурсов одного процессора. Поток представляет собой последовательность инструкций, которые могут быть выполнены параллельно с другими потоками в рамках одного процесс~\cite{threads}.

Процесс --- это программа в состоянии выполнения. Когда программа или приложение запускается, создается процесс. Процесс может состоять из одного или нескольких потоков. Каждый поток представляет собой сегмент процесса, который выполняет задачи, стоящие перед приложением. Процесс завершается, когда все его потоки завершают свою работу.

В многопоточных программах необходимо учитывать, что если потоки запускаются последовательно и передают управление друг другу, то не получится полностью использовать потенциал многопоточности и получить выигрыш от параллельной обработки задач. Чтобы достичь максимальной эффективности, потоки должны выполняться параллельно, особенно для независимых по данным задач.

Одной из проблем, возникающих при использовании многопоточности, является совместный доступ к данным. Возникает конфликт, когда два или более потоков пытаются записать в одну и ту же ячейку памяти одновременно. Для решения этой проблемы используется механизм синхронизации доступа к данным, такой как мьютекс (mutex) или блокировка. Мьютекс позволяет одному потоку работать с данными в монопольном режиме, пока другие потоки ожидают освобождения мьютекса.

Критическая секция --- это набор инструкций, выполняемых между захватом и освобождением мьютекса. Во время захвата мьютекса другие потоки, которым требуется доступ к общим данным, должны ждать его освобождения. Поэтому необходимо минимизировать объем кода в критической секции, чтобы другие потоки могли получить к ней доступ как можно быстрее.

Таким образом, использование многопоточности в программировании требует хорошего понимания концепций синхронизации и разработки эффективного кода для доступа к общим данным, чтобы достичь максимальной производительности и избежать состояний гонки.

\section{Дерево синтаксических зависимостей \\
	текста}

Дерево синтаксических зависимостей представляет собой структуру, которая отображает отношения между словами в предложении или тексте. Оно является графическим представлением синтаксической структуры предложения и позволяет анализировать связи между словами.

В дереве синтаксических зависимостей каждое слово представлено узлом, а связи между словами представлены направленными ребрами. Узлы могут иметь различные свойства, такие как лемма (нормальная форма слова), часть речи, морфологические характеристики и другие.

Каждое ребро в дереве синтаксических зависимостей указывает на отношение между словами. Например, ребро может указывать на то, что слово является зависимым от другого слова в качестве подлежащего, сказуемого, прямого или косвенного дополнения и т.д. Все эти отношения образуют иерархическую структуру, которая отображает семантику и синтаксис в предложении.

Построение дерева синтаксических зависимостей основано на лексическом и синтаксическом анализе текста. При анализе предложения или текста, алгоритм разбивает его на токены (слова), определяет их часть речи, а затем выстраивает связи между словами, чтобы создать дерево синтаксических зависимостей.

Дерево синтаксических зависимостей полезно для понимания семантики предложения, разрешения синтаксической структуры, а также для выполнения различных задач в области обработки естественного языка, таких как извлечение информации, ответы на вопросы и машинный перевод.

Библиотека spacy предоставляет мощные инструменты для анализа синтаксических зависимостей и визуализации деревьев синтаксических зависимостей, которые помогают исследователям и разработчикам в понимании структуры текста и автоматическом анализе естественного языка.

\section{Вывод}

В данном разделе были рассмотрены понятия многопоточности и деревьев синтаксических зависимостей в тексте.