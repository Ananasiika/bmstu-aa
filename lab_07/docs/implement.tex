\chapter{Технологическая часть}

В данном разделе будут рассмотрены средства реализации, а также представлены листинги алгоритма Кнута-Морриса-Пратта и его модификации с использованием эвристики <<плохого>> символа.

\section{Средства реализации}
В данной работе для реализации был выбран язык программирования $Python \cite{python-lang}$. 
В текущей лабораторной работе требуется замерить процессорное время для выполняемой программы, а также построить графики. 
Все эти инструменты присутствуют в выбранном языке программирования.

Время было замерено с помощью функции \textit{process\_time(...)} из библиотеки $time \cite{python-lang-time}$.


\section{Реализация алгоритмов}

В листинге \ref{lst:lst1} представлен алгоритм для определения наибольшей длины суффикса подстроки, являющегося ее префиксом, в листинге \ref{lst:knt} --- алгоритм Кнута-Морриса-Пратта, а в листинге \ref{lst:bm} его модификация с использованием эвристики <<плохого>> символа.
\clearpage

\begin{center}
    \captionsetup{justification=raggedright,singlelinecheck=off}
    \begin{lstlisting}[label=lst:lst1,caption=Алгоритм определения наибольшей длины суффикса подстроки]
def create_lps(pattern):
	m = len(pattern)
	lps = [0] * m
	length = 0
	i = 1
	while i < m:
		if pattern[i] == pattern[length]:
			length += 1
			lps[i] = length
			i += 1
		else:
			if length != 0:
				length = lps[length - 1]
			else:
				lps[i] = 0
				i += 1
return lps
\end{lstlisting}
\end{center}


\begin{center}
    \captionsetup{justification=raggedright,singlelinecheck=off}
    \begin{lstlisting}[label=lst:knt,caption=Алгоритм Кнута-Морриса-Пратта]
def kmp_search(text, pattern):
	n = len(text)
	m = len(pattern)
	lps = create_lps(pattern)
	indices = -1
	i, j = 0, 0
	while i < n:
		if pattern[j] == text[i]:
			i += 1
			j += 1
		if j == m:
			indices = i - j
			break
		else:
			if j != 0:
				j = lps[j - 1]
			else:
				i += 1
return indices
\end{lstlisting}
\end{center}




\begin{center}
    \captionsetup{justification=raggedright,singlelinecheck=off}
    \begin{lstlisting}[label=lst:bm,caption=Модифицированный алгоритм с использованием эвристики <<плохого>> символа]
def bm_search(text, pattern):
	n = len(text)
	m = len(pattern)
	bad_char = {}
	indices = -1
	for i in range(m):
		bad_char[pattern[i]] = i
	i = 0
	while i <= n - m:
		j = m - 1
		while j >= 0 and pattern[j] == text[i+j]:
			j -= 1
		if j < 0:
			indices = i
			break
		else:
			if text[i+j] in bad_char:
				i += max(1, j - bad_char[text[i+j]])
			else:
				i += max(1, j)
	return indices
\end{lstlisting}
\end{center}


\section{Сведения о модулях программы}
Программа состоит из двух модулей:
\begin{itemize}
	\item \textit{main.py} - файл, содержащий меню программы, а также весь служебный код;
    \item \textit{algorithms.py} - файл, содержащий реализацию рассматриваемых алгоритмов поиска подстроки в строке.
\end{itemize}


\section{Функциональные тесты}

В таблице \ref{tbl:functional_test} приведены тесты для функций программы.
Тесты \textit{для всех функций} пройдены успешно.

Если функция находит подстроку в строке, то она возвращает индекс первого символа подстроки в строке, иначе она возвращает -1.


\begin{center}
	\begin{threeparttable}
    \captionsetup{justification=raggedright,singlelinecheck=off}
    \caption{Функциональные тесты}
    \label{tbl:functional_test}
    \begin{tabular}{|c|c|c|c|} 
    \hline
	Подстрока & Строка & Полученный индекс & Ожидаемый индекс \\ \hline
	test & testik & 0 & 0 \\ \hline
	est & testik & 1 & 1 \\ \hline
	one & testikone & 6 & 6 \\ \hline
	g & testikonesecond & -1 & -1 \\ \hline
	\end{tabular}
	\end{threeparttable}
\end{center}

\section{Вывод}

Были представлены листинги всех алгоритмов --- Кнута-Морриса-Пратта и его модификации с использованием эвристики <<плохого>> символа. 
Также в данном разделе была приведена информации о выбранных средствах реализации и сведения о модулях программы, проведено функциональное тестирование.
