\chapter*{Заключение}
\addcontentsline{toc}{chapter}{Заключение}

В результате эксперимента было получено, что худший случай для этих алгоритмов --- отсутствие подстроки в строке.
Реализация алгоритма Кнута-Морриса-Прата работает примерно в 1.5 раза быстрее реализации модифицированного алгоритма, но при этом использует примерно в 2 раза больше операций сравнения.  

Цель, которая была поставлена в начале лабораторной работы была достигнута, а также в ходе выполнения лабораторной работы были решены следующие задачи:

\begin{itemize}[label=---]
    \item описаны алгоритмы решения задачи поиска подстроки в строке -- Кнута-Морриса-Пратта и его модификацию с использованием эвристики <<плохого>> символа;
	\item реализованы эти алгоритмы;
	\item проведен сравнительный анализ рассматриваемых алгоритмов по времени и по количеству сравнений;
	\item подготовлен отчет по лабораторной работе.
\end{itemize}