\chapter*{Введение}
\addcontentsline{toc}{chapter}{Введение}

Поиск подстроки является важной операцией в области обработки текстов и анализа данных. 
Он широко используется в различных приложениях, включая поисковые системы, обработку естественного языка, биоинформатику и многое другое. 
Поэтому понимание и применение оптимальных алгоритмов поиска подстроки играют важную роль в разработке эффективных приложений~.

\textbf{Целью данной работы} является исследование алгоритма Кнута-Морри-
са-Пратта и его модификацию с использованием эвристики <<плохого>> символа.
Для достижения поставленной цели необходимо выполнить следующие задачи:
\begin{itemize}[label=---]
    \item описать алгоритмы решения задачи поиска подстроки в строке -- Кнута-Морриса-Пратта и его модификацию с использованием эвристики <<плохого>> символа;
    \item реализовать эти алгоритмы;
    \item провести сравнительный анализ рассматриваемых алгоритмов по времени и по количеству сравнений;
    \item подготовить отчет по лабораторной работе.
\end{itemize}
