\chapter{Аналитическая часть}
В этом разделе будет представлена информация об алгоритме Кнута-Морриса-Пратта и его модификации с использованием эвристики <<плохого>> символа.

\section{Алгоритм Кнута-Морриса-Пратта}

Алгоритм Кнута-Морриса-Пратта (КМП) --- это эффективный алгоритм для поиска подстроки в строке. 
Он был разработан Дональдом Кнутом и Валтором Моррисом в 1970-х годах, а затем усовершенствован Джеймсом Праттом.

Основная идея алгоритма КМП заключается в использовании префиксной функции, которая предварительно вычисляется для искомой подстроки. 
Префиксная функция определяет наибольшую длину суффикса подстроки, являющегося ее префиксом. 
Это позволяет алгоритму оптимально смещаться по строке в случае несоответствия символов, минимизируя количество сравнений~\cite{analyt}.

Процесс работы алгоритма КМП включает следующие шаги:
\begin{enumerate}[label=\arabic*)]
	\item Вычисление префиксной функции для искомой подстроки.
	Для каждой позиции в подстроке определяется длина наибольшего общего префикса и суффикса. Эти значения записываются в отдельный массив.
	\item Поиск подстроки в строке.
	Алгоритм просматривает символы строки и подстроки поочередно. 
	При сравнении символов, если они не совпадают, используется префиксная функция для оптимального смещения указателей. 
	Если совпадение найдено, алгоритм продолжает сравнивать следующие символы.
\end{enumerate}

Алгоритм имеет линейную сложность и работает за $O(n + m)$, где $n$ --- длина строки, а $m$ --- длина подстроки.

\section[Модифицированный алгоритм с использованием эвристики <<плохого>> символа]{Модифицированный алгоритм с \\использованием эвристики <<плохого>> \\символа}

Модифицированный алгоритм поиска подстроки с использованием эвристики <<плохого>> символа является улучшенной версией классического алгоритма Бойера-Мура. 
Он был разработан для повышения эффективности поиска и сокращения количества сравнений.

Основная идея модифицированного алгоритма заключается в использовании таблицы <<плохого>> символа. 
Эта таблица предоставляет информацию о смещении, которое можно сделать при несоответствии символов в процессе поиска. 
При каждом несовпадении символов, алгоритм консультируется с таблицей <<плохого>> символа и смещает указатель по тексту на максимально возможное количество позиций, основываясь на информации из таблицы.

Таблица <<плохого>> символа строится на основе позиций символов в подстроке. 
Для каждого символа в подстроке вычисляется смещение, которое необходимо выполнить в случае его несоответствия с символом в тексте. 
Если символ не содержится в подстроке, то смещение равно длине подстроки.

В результате использования эвристики <<плохого>> символа, алгоритм сокращает количество проверок и сравнений, пропуская часть текста, где не может находиться искомая подстрока. 
Это позволяет значительно повысить производительность поискового алгоритма.

\section{Вывод}

В данном разделе был рассмотрен алгоритм Кнута-Морриса-Пратта и его модификация с использованием эвристики <<плохого>> символа.